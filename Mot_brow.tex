\section{Nascita moto Browniano}
Iniziamo considerando il cammino casuale unidimensionale simmetrico $p=1/2$. Vogliamo ora adottare una nuova prospettiva sul problema, vederlo più da "\textit{lontano}" riscalandolo tramite intervalli di tempo e spazio sempre più piccoli, infinitesimi. Passando poi al limite otteremo un primo esempio di moto Brownian.

Sia $X(t)$ la posizione al tempo $t$:
\[X(t)=\Delta x\left(X_1+\dots+X_{\lfloor \frac{t}{\Delta t} \rfloor}\right)\]
Dove $X_i$ vale più o meno 1 a seconda che il passo $\Delta x$ sia a destra o a sinistra. Tutto il processo è riscalato ad intervallini $\Delta t$. Dunque si ottiene:
\begin{itemize}
    \item $\mathbb{E}X(t)=0$
    \item $\mathbb{V}arX(t)=\left(\Delta x\right)^2\lfloor \frac{t}{\Delta t} \rfloor$
\end{itemize}

Ma se gli intervalli di spazio e tempo tendono a zero con lo stesso ordine si ottiene una varianza nulla, e dunque il processo si schiaccia sull'asse $x$. Non è evidentemente l'obiettivo di una buona modellizzazione. 

Einstein, in un celebre articolo del 1905, "suggerisce" un approccio diverso, nel quale tra spazio e tempo interrcorre una relazione del tipo:
\[\Delta x=\sigma\sqrt{\Delta t}\]
Così facendo si ottiene, tramite il teorema del limite centrale otteniamo:
\[X(t)\sim N(0,\sigma^2t)\]
Ecco che otteniamo un moto Browniano. 

Passiamo a una definizione rigorosa.

\begin{definition}
Un processo stocastico si definisce \textbf{moto Browniano} se:
\begin{enumerate}
    \item $X(0)=0$
    \item Il processo ha incrementi indipendenti e stazionari
    \item $\forall t>0$ e per un parametro $\sigma^2$ si ha:
    \[X(t)\sim N(0,\sigma^2t)\]
    \item Le traiettorie del processo sono continue
\end{enumerate}
\end{definition}
In realtà la condizione (4.) può essere dimostrata dalle altre. Un moto Browniano ha la peculiare proprietà di essere continuo ovunque ma non differenziabile in nessun punto.

La condizione (3.) con la stazionarietà degli incrementi è equivalente a: 
\[X(t)-X(s)\sim N(0,\sigma^2(t-s)) \hspace{7px} \forall t,s\]
La condizione (3.) ci fornisce inoltre un importantissima informazione, la distribuzione del moto in un dato momento, in particolare:
\[f_t(x)=\frac{1}{\sqrt{2\pi\sigma^2 t}}e^{-\frac{x^2}{2\sigma^2t}}\]

Osservando che:
\[\{X(t_1)=x_1,X(t_2)=x_2,\dots,X(t_n)=x_n\}=\]
\[\{X(t_1)=x_1-\underbrace{x_0}_{=0},X(t_2)-X(t_1)=x_2-x_1,\dots,X(t_n)-X(t_{n-1})=x_n-x_{n-1}\}\]
Posso calcolare la densità congiunta (di un moto Browniano tandard, cioè con $\sigma=0$), ricordandomi dell'indipendenza degli intervalli disgiunti:
\[f(x_1,\dots,x_n)=\frac{1}{(2\pi)^{\frac{n}{2}}[t_1\dots(t_n-t_{n-1})]}e^{-\frac{1}{2}\left[\frac{x_1^2}{t_1}+\dots+\frac{(x_n-x_{n-1})^2}{t_n-t_{n-1}}\right]}\]

\begin{example}[Ponte Browniano]
Dato un moto browniano standard $\{X(t)\}$ e un $t\in\mathbb{R}_+$ fissato tale che $X(t)=B$, per $s<t$ calcolo:
\[f_{s|t}(x|B)=\frac{f_{s,t}(x,B)}{f_{s}(x)}=\frac{f_s(x)f_{t-s}(x-B)}{f_t(B)}=K_1exp\left\{-\frac{x^2}{2s}-\frac{(x-B)^2}{2(t-s)}\right\}\]
\[=K_2exp\left\{-x^2\left(\frac{1}{2s}+\frac{1}{2(t-s)}\right)+\frac{xB}{t-s}\right\}=K_3exp\left\{\frac{-t}{2s(t-s)}\left(x^2-\frac{2xsB}{t}\right)\right\}\]
\[=K_4exp\left\{\frac{-t}{2s(t-s)}\left(x-\frac{s}{t}B\right)^2\right\}\]
($K_i$ sono indipendenti da $s$).
\newline Ottenendo dunque che la distribuzione di probabilità in un ponte browniano è una normale di media e varianza:
\begin{itemize}
    \item $\mathbb{E}[X(s)|X(t)=B]=\frac{s}{t}B$
    \item $\mathbb{V}ar[X(s)|X(t)=B]=\frac{s}{t}(t-s)$
\end{itemize}
\end{example}

Altro esempio degno di nota, che richiama la rovina del giocatore (\ref{Rov_gioc}) è il seguente.

\begin{example}
Sia $T_a=Inf\{t>0:X(t)\geq a\}$ e studiamone la distribuzione:
\[\mathbb{P}(X(t)\geq a)=\mathbb{P}(X(t)\geq a|T_a\leq t)\mathbb{P}(T_a\leq t)+\mathbb{P}(X(t)\geq a|T_a>t)\mathbb{P}(T_a>t)\]
Si nota subito che:
\[\mathbb{P}(X(t)\geq a|T_a > t)=0\]
E inoltre:
\[\mathbb{P}(X(t)\geq a|T_a\leq t)=\frac{1}{2}\]
Questo secondo risultato deriva dal fatto che per simmetria ogni traiettoria sopra ad $a$ ce n'è una sotto.

Si ottiene così:
\[\mathbb{P}(T_a\leq t)=2\mathbb{P}(X(t)\geq a)\]
Dunque:
\[f_{T_a}(t)=\frac{d}{dt}\left[2\int_a^{+\infty}\frac{1}{\sqrt{2\pi\sigma^2 t}}e^{-\frac{x^2}{2\sigma^2t}} \,dx\right]\underset{y=\frac{x}{\sqrt{\sigma^2t}}}{=}\frac{d}{dt}\left[\sqrt{\frac{2}{\pi}}\int_{a/\sqrt{\sigma^2t}}^{+\infty}e^{-\frac{y^2}{2}} \,dy\right]=\]
\[\frac{a}{\sqrt{2\pi\sigma^2t^3}}e^{\frac{-a^2}{2\sigma^2t}}\]
\end{example}
 Da questo risultato si può calcolare la distribuzione di $M:=max_{s=0,\dots,t}\{X(s)\}$. Infatti:
 \[\frac{d}{da}\mathbb{P}(M\geq a)=\frac{d}{da}\left(\mathbb{P}(T_a\leq t)\right)=\frac{d}{da}\left(2\int_a^{+\infty}\frac{1}{\sqrt{2\pi\sigma^2 t}}e^{-\frac{x^2}{2\sigma^2t}} \,dx\right)=\sqrt{\frac{2}{\pi\sigma^2t}}e^{-\frac{a^2}{2\sigma^2t}}\]
\subsection{Tipologie di Moti Browniani}
\begin{definition}
Un processo stocastico $\{X(t)\}$ è un \textbf{moto browniano con drift} se:
\begin{itemize}
    \item $X(0)=0$
    \item Ha incrementi stazionari e indipendenti
    \item $X(t)\sim N(\mu t,\sigma^2 t)$
\end{itemize}
\end{definition}
Un moto browniano con drift può anche essere visto come la composizione lineare di un moto browniano standard e un drift $\mu t$ del tipo:
\[X(t)=\sigma B(t)+\mu t\]

\begin{definition}
Dato un moto browniano con drift $\{Y(t)\}$ e coefficiente di drift $\mu$ si definisce \textbf{moto browniano geometrico} il seguente:
\[X(t)=e^{Y(t)}\]
\end{definition}

Calcoliamo ora il valore atteso di un moto browniano geometrico dat la sua storia siano ad un tempo stabilito:
\[\mathbb{E}[X(t)|X(u), 0\leq u\leq s] \hspace{25px} \text{per} \hspace{5px} s<t\]
Per il calcolo si noti che conoscere $Y(u)$ o conoscere $e^{Y(u)}$ è la stessa cosa.
\[\mathbb{E}[X(t)|X(u), 0\leq u\leq s]=\mathbb{E}[e^{Y(t)}|Y(u), 0\leq u\leq s]=X(s)\mathbb{E}[e^{Y(t)-Y(s)}|Y(u), 0\leq u\leq s]\]
Ricordando che la funzione caratteristica di una normale è della forma $e^{\mu t+\frac{\sigma^2t^2}{2}}$ (\ref{char_N}):
\[\mathbb{E}[X(t)|X(u), 0\leq u\leq s]=X(s)e^{\mu(t-s)+\frac{\sigma^2}{2}(t-s)}\]
Se $\mu=-\frac{\sigma^2}{2}$ si ottiene la \href{https://en.wikipedia.org/wiki/Martingale_(probability_theory)}{proprietà martingale}.

Come ultimo spunto siano ad esempio $X_n$ i prezzi di un'azione al tempo $n$, sembra ragionevole considerare la variazione percentuale $Y_n=\frac{X_n}{X_{n-1}}$ come IID. Si verifichi (pag. 645 - S.M.Ross) che $\{X_n\}$ è un moto browniano geometrico (\textit{Hint: si scriva $X_n$ in funzione delle $Y_i$ e si usi il teorema ddel limite centrale}).