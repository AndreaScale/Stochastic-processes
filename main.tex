\documentclass{book}
\usepackage[utf8]{inputenc}
\usepackage[english]{babel}
\usepackage{amsthm}
\usepackage{amsfonts}
\usepackage{subfiles}
\usepackage{graphicx}
\usepackage[scr]{rsfso}
\usepackage{hyperref}
\usepackage{mathtools}
\usepackage{array,amsfonts}
\usepackage{magaz}
\usepackage{amssymb}
\usepackage[shortlabels]{enumitem}
\usepackage{lipsum}
\usepackage[dottedtoc]{classicthesis}
\usepackage{tocloft}
\usepackage{cancel}
\usepackage{movie15}
\usepackage{blindtext}
\usepackage{hyperref}
\usepackage{graphicx}
\usepackage{animate}

%-----------------------------------------------------------------------------------------------
\def\@tocline#1#2#3#4#5#6#7{\relax
  \ifnum #1>\c@tocdepth % then omit
  \else
    \par \addpenalty\@secpenalty\addvspace{#2}%
    \begingroup \hyphenpenalty\@M
    \@ifempty{#4}{%
      \@tempdima\csname r@tocindent\number#1\endcsname\relax
    }{%
      \@tempdima#4\relax
    }%
    \parindent\z@ \leftskip#3\relax \advance\leftskip\@tempdima\relax
    \rightskip\@pnumwidth plus4em \parfillskip-\@pnumwidth
    #5\leavevmode\hskip-\@tempdima #6\nobreak\relax
    \dotfill\hbox to\@pnumwidth{\@tocpagenum{#7}}\par
    \nobreak
    \endgroup
  \fi}
  
\newcommand\myfunc[5]{%
  \begingroup
  \setlength\arraycolsep{0pt}
  #1\colon\begin{array}[t]{c >{{}}c<{{}} c}
             #2 & \to & #3 \\ #4 & \to & #5 
          \end{array}%
  \endgroup}
\newtheorem{theorem}{Theorem}
\newtheorem{corollary}{Corollary}[theorem]
\newtheorem{lemma}[theorem]{Lemma}
\theoremstyle{remark}
\newtheorem{proposition}{Proposition}[section]
\theoremstyle{proposition}
\newtheorem*{remark}{Remark}
\theoremstyle{definition}
\newtheorem{definition}{Definition}[section]
\newtheorem{example}{Example}[section]
\newtheorem{observation}{Observation}[section]
\usepackage{tocloft}
\renewcommand{\cftchapdotsep}{\cftdotsep}

\begin{document}
\title{Appunti di Calcolo delle Probabilità 2}
\author{Andrea Scalenghe}
\date{Settembre2021}
\begin{titlepage}
\maketitle
\end{titlepage}
\pdfbookmark[1]{\contentsname}{tableofcontents}
\setcounter{tocdepth}{2} % <-- 2 includes up to subsections in the ToC
\setcounter{secnumdepth}{3} % <-- 3 numbers up to subsubsections
\manualmark
\markboth{\spacedlowsmallcaps{\contentsname}}{\spacedlowsmallcaps{\contentsname}}
\tableofcontents
\automark[section]{chapter}
\renewcommand{\chaptermark}[1]{\markboth{\spacedlowsmallcaps{#1}}{\spacedlowsmallcaps{#1}}}
\renewcommand{\sectionmark}[1]{\markright{\thesection\enspace\spacedlowsmallcaps{#1}}}
\newpage
\section{Premessa}
Si presentano qui degli appunti sintetici del corso di Calcolo delle probabiltà 2 tenuto nell' a.a. 2021/2022 nel Dipartimento di Matematica dell'Università di Torino dalla Professoressa Laura Sacerdote.

Il corso di \href{https://www.matematica.unito.it/do/corsi.pl/Show?_id=z1b7}{CP2} si appoggia sulle conoscenze acquisite dagli studenti tramite l'insegnamento di \href{https://www.matematica.unito.it/do/corsi.pl/Show?_id=hffu}{CPS}. Si daranno dunque pregresse delle conoscenze in queste dispense. Nel caso fosse necessario rivedere dei concetti o dimostrazioni si rimanda alla dispense di CPS.
\part{Intro}
\chapter{Inizio}
\section{Elementi di calcolo delle probabilità}
\subsection{Funzione generatrice congiunta}
Si inizia introducendo un concetto non affrontato nel corso di CPS dell'a.a 2020/2021, la \textit{funzione generatrice congiunta}.

\begin{definition}
Date $(X_i)$ per $i=1,...,n$ vettore aleatorio si definisce \textbf{funzione generatrice congiunta} la seguente:

    \begin{alignat}{4}
\Phi:\; && \mathbb{R}^n && \;\to\;     & \mathbb{R} \\
     && \underline{t}    && \;\mapsto\; & \Phi(\underline{t})
\end{alignat}
Dove $\Phi(\underline{t})=\mathbb{E}[exp(\sum_{i=1}^nt_iX_i)]$.
\end{definition}

Come esercizio si può calcolare la normale multivariata di $(X_i)$ per $i=1,...,n$ dove $X_i=\sum_{j=1}^na_jZ_j+b_i$ con $Z_j$ gaussiane standard $\forall j=1,...,m$.

\subsection{Non esistenza della funzione di densità}
Osserviamo una v.a. $X$ normale. La sua funzione caratteristica è:
\begin{equation} \label{char_N}
    \Phi(t)=e^{t\mu+\frac{t^2\sigma^2}{2}}
\end{equation}
Mentre la sua funzione di distribuzione è:
\begin{equation}
    f_X(t)=\frac{1}{\sqrt{2\pi \sigma^2}}e^{\frac{(t-\mu)^2}{2\sigma^2}}
\end{equation}
Osserviamo che se facciamo tendere a zero la varianza ($\sigma^2\rightarrow 0$) la funzione caratteristica rimane definita ($\Phi(t)=e^{t\mu}$), mentre la funzione di densità risulta indefinita. In particolare, quando la varianza tende a zero, tutta la massa di probabilità si concentra in un punto, situazione descritta dalla \href{https://en.wikipedia.org/wiki/Dirac_delta_function}{Delta di Dirac}.

\begin{definition}
Si definisce \textbf{tasso di rottura} (\textit{faliure/hazard rate}) di una variabile aletoria $X$ la funzione:
\[r(t)=\frac{f_x(t)}{\overline{F}_X(t)}\]
\end{definition}

Dato il tasso di rottura di una v.a. possiamo ricondurci alla sua distribuzione:
\[r(t)=\frac{f_x(t)}{\overline{F}_X(t)}=\frac{\frac{d}{dt}\left(F_X(t)\right)}{1-F_X(t)}\]
Quindi:
\[\frac{1}{1-F}dF=r(t)dt \Rightarrow \int_0^t\frac{1}{1-F}dF=\int_0^t r(t)dt \Rightarrow -ln(1-F(t))=\int_0^t r(t)dt\]
Ottenendo:
\[F_X(t)=1-e^{-\int_0^t r(t)dt}\]
\subsection{Condizionamenti}

Poste $X,Y$ v.a. ho che:
\begin{equation}
    \mathbb{P}(X|Y=y)
\end{equation}
è una distribuzione di probabilità. Infatti:
\begin{itemize}
    \item $\sum_x \mathbb{P}(X=x|Y=y)= \frac{\sum_x \mathbb{P}(X=x,Y=y)}{\mathbb{P}(Y=y)}=\frac{\mathbb{P}(Y=y)}{\mathbb{P}(Y=y)}=1$
    \item $\mathbb{P}(X=x|Y=y)\geq0$
\end{itemize}
Si può quindi definire il valore atteso, detto \textbf{attesa condizionata}:
\begin{equation}
    \mathbb{E}(X|Y=y)=\sum_{x}x\mathbb{P}(X=x|Y=y)
\end{equation}
Si osserva immediatamente che l'attesa condizionata è una funzione di y: $\varphi(y)$.
Per esercizio si può calcolare l'attesa condizionata per due Poisson di parametri distinti o per due binomiali di stesso parametro (IID).

Nel continuo si ha un problema nella definzione della funzione di probabilità, poichè seguendo la definizione di probabilità condizionata si ha uno zero al denominatore (dato da $\mathbb{P}(Y=y)=0$). Per definire la funzione di probabilità ci serviamo quindi di un intorno di $y$ e applichiamo successivamente il teorema di de l'Hôpital.

    \[\mathbb{P}(X\leq x|Y\in[y,y+h])=\frac{\mathbb{P}(X\leq x,Y\in[y,y+h])}{\mathbb{P}(Y\in[y,y+h])}=
    \frac{\int_{-\infty}^x\int_y^{y+h}f_{X,Y}(x,y) \, dxdy}{\int_y^{y+h}f_Y(y) \,dy}\]
Applicando dunque de l'Hôpital si ottiene:
    \[\lim_{h\to 0}\mathbb{P}(X\leq x|Y\in[y,y+h])=\frac{\int_{-\infty}^xf_{X,Y}(x,y) \,dx}{f_Y(y)}=\int_{-\infty}^x\frac{f_{X,Y}(x,y)}{f_Y(y)} \, dx\]
Osservando che per la continuità a destra della funzione di proabilità vale:
\[\mathbb{P}(x\leq x|Y=y)=\lim_{h\to0}\mathbb{P}(X\leq x|Y\in[y,y+h])\]
Si ottiene che la funzione di ripartizione di $X$ condizionata a $Y=y$:
\begin{equation}
    f_{X|Y}(x,y)=\frac{f_{X,Y}(x,y)}{f_Y(y)}
\end{equation}
Si verifica facilmente che si tratta di una funzione di ripartizione. E' inoltre ovvia l'estesione dell'attesa condizionata al caso continuo. Per esercizio si può calcolare l'attesa condizionata di $X$ rispetto a $Y=y$ nei due seguenti casi:
\begin{itemize}
    \item $f_{X,Y}(x,y)=6xy(2-x-y)$ per $(x,y)\in(0,1)\times(0,1)$
    \item $f_{X,Y}(x,y)=\frac{1}{2}ye^{-xy}$ per $y\in(0,2) , x\in\mathbb{R}_+$
\end{itemize}

Date due varibili aleatorie $X,Y$ ci concentriamo ora sul significato dell'oggetto:
\begin{equation} \label{E(X|Y)}
    \mathbb{E}(X|Y)
\end{equation}
Ci ricordiamo che $\mathbb{E}(X|Y=y)=\varphi(y)$ è una funzione di $y$, che conserva le ipotesi di misurabilità. Dunque $\mathbb{E}(X|Y)$ è una \textbf{variabile aleatoria}, il cui supporto è l'immagine attraverso $\varphi$ del supporto di $Y$ ($\varphi(S_Y)$). Un fondamentale teorema che ci viene in contro, e che ci fa capire l'importanza della variabile aleatoria costruita in \ref{E(X|Y)} è il seguente.

\newcommand{\E}{\mathbb{E}}
\newcommand{\V}{\mathbb{V}ar}

\begin{theorem}[Doppia attesa]
Siano $X,Y$ v.a. tali che $\mathbb{E}X$ esista, allora:
\begin{equation}
    \mathbb{E}X=\mathbb{E}[\mathbb{E}(X|Y)]
\end{equation}
\begin{proof}
Caso discreto:
\[ \E[\E(X|Y)]=\sum_y \E(X|Y=y)\mathbb{P}(Y=y)=\sum_y\sum_x x\mathbb{P}(X=x|Y=y)\mathbb{P}(Y=y)\]
\[=\sum_x x\sum_y\mathbb{P}(X=x,Y=y)=\sum_x x\mathbb{P}(X=x)=\E X \]
Caso continuo:
\[ \E[\E(X|Y)]=\int_{\mathbb{R}}\E(X|Y=y)f_Y(y) \,dy=\int\int_{\mathbb{R}^2}x f_{X|Y}(x,y)f_Y(y) \,dxdy \]
\[\int_{\mathbb{R}}x \int_{\mathbb{R}}f_{X,Y}(x,y) \,dy \,dx=\int_{\mathbb{R}}x f_X(x) \,dx =\E X \]
\end{proof}
\end{theorem}

Si provino i seguenti come esercizi:
\begin{itemize}
    \item Calcolare il valore atteso di una geometrica condizionando (il valore atteso) con una v.a: $$Y= 
    \begin{cases}
    0 & \text{Primo esito insuccesso}
    \\ 1 & \text{Primo esito successo}
    \end{cases}$$
    Tenendo presente che il valore atteso, sapendo che il primo esito è un \textit{insuccesso}, è il valore atteso della v.a. più 1.
    \item Ci sono 3 porte:
    \begin{enumerate}
        \item Ci vogliono 2 ore e si è salvi
        \item Ci voglio 3 ore e si torna alle porte
        \item Ci volgio 5 ore e si torna alle porte
    \end{enumerate}
    Ogni volta che si torna alle porte non ci si ricorda la scelta precendete. Qual è il tempo medio di attesa per uscire. La scelta della porta è casuale ($p=\frac{1}{3}$). (Si consiglia di condizionare sulla scelta della porta. \textit{Il risultato è 10 ore}).
\end{itemize}

Ci poniamo invece ora in una situzione comune, siano $(X_i)$ v.a. IID con $\E X_i=\E X$ e $N$ v.a. Poniamo $Y=\sum_{i=0}^NX_i$ e calcoliamone il valore atteso.

\[\E Y=\E[\E(\sum_{i=0}^N X_i|N)]=\sum_n\E(\sum_{i=0}^N X_i|N=n)\mathbb{P}(N=n)=\sum_n\E(\sum_{i=0}^n X_i)\mathbb{P}(N=n)\]
\[\sum_n \sum_{i=0}^n\E X_i \cdot\mathbb{P}(N=n)=\sum_n n\cdot\E X\cdot \mathbb{P}(N=n)=\E X\cdot\E N\]
Si lascia da verificare che $\mathbb{V}arY = \mathbb{V}arX\cdot\E N+\E^2X\cdot\mathbb{V}arN$.

\begin{theorem}\label{Varr}
Siano $X,Y$ v.a. vale:
\begin{equation}
    \V X=\E[\V(X|Y)]+\V[\E(X|Y)]
\end{equation}
\begin{proof}
Si noti che, come per l'attesa condizionata anche $\V (X|Y)$ è una variabile aleatoria. Ha quindi senso calcolarne il valore atteso. Ora il primo addendo si scompone in:
\[\E[\V(X|Y)]=\E[\E(X^2|Y)-\E^2(X|Y)]=\E X^2-\E[\E^2(X|Y)]\]
Mentre il secondo addendo:
\[\V[\E(X|Y)]=\E[\E^2(X|Y)]-\E^2[\E(X|Y)]=\E[\E^2(X|Y)]-\E^2(X)\]
Sommando quindi i due addendi segue la tesi.
\end{proof}
\end{theorem}
\subsection{Mancanza di memoria}
Ricordiamo che la v.a. esponenziale gode della propietà di \textit{mancanza di memoria}, o e \textit{priva di usura}. Vale a dire, sia $X\sim Exp(\lambda)$:
\begin{equation} \label{Manc_mem}
\mathbb{P}(X>t+r|X>r)=\mathbb{P}(X>t)
\end{equation}
$\forall t,r\in\mathbb{R}$. 

La condizione \ref{Manc_mem} equivale a:
\[\mathbb{P}(X>t+s)=\mathbb{P}(X>t)\mathbb{P}(X>s)\]
\begin{proposition}
La distribuzione esponenziale è l'unica distribuzione continua priva di usura.
\begin{proof}
Sia $X$ v.a. e $\overline{F}_X(x)=\mathbb{P}(X>x)$. Chiamiamo $g:=\overline{F}$, e otteniamo: $g(x+y)=g(x)g(y)$ , $\forall x,y\in\mathbb{R}$.
Si dimostra per induzione che:
    \[g\left(\frac{m}{n}\right)=\left[g\left(\frac{1}{n}\right)\right]^m\]
Ora:
\[g(1)=g\left(\frac{n}{n}\right)=\left[g\left(\frac{1}{n}\right)\right]^n\]
Ottenendo: \label{Eq_dim_exp}
\[g\left(\frac{m}{n}\right)=g(1)^{\frac{m}{n}}\]
Per la continuità a destra della funzione comulativa di probabilità:  \[g(x)=g(1)^x\] $\forall x\in\mathbb{R}$.
Poichè $g(1)=g^2\left(\frac{1}{2}\right)\geq 0$ posso considerare:
\[\lambda=-ln(g(1))\]
E ottendo così:
\[F_X(x)=1-\overline{F}_X(x)=1-g(x)=1-e^{-\lambda x}\]
Dunque se $X$ è priva di usura è necesssariamente un'esponenziale. 
\end{proof}
\end{proposition}
 Ulteriori proprietà dell'esponenziale sono presenttate in quanto segue:
 \begin{itemize}
     \item $X_i\sim Exp(\lambda)$ IID $\Rightarrow \sum_{i=0}^nX_i\sim\Gamma(\lambda,n)$
     \item $X_1\sim Exp(\lambda_1)$ e $X_2\sim Exp(\lambda_2)$ allora $\mathbb{P}(X_1<X_2)=\frac{\lambda_1}{\lambda_1+\lambda_2}$ (Si prova condizionando su $X_1$)
     \item $X_i\sim Exp(\lambda_i)$ indipendenti allora $min\{X_i\}\sim Exp\left(\sum_i \lambda_i\right)$ (Si prova con $\{min\{X_i\}>x\}=\{X_1>x,\dots,X_n>x\}$)
     \item $\mathbb{P}(X_i=min\{X_j\})=\frac{\lambda_i}{\sum_j \lambda_j}$ (Si prova con $\{X_i=min\{X_j\}\}=\{X_i<min_{j\neq i}\{X_j\}\}$)
 \end{itemize}

\part{Processi stocastici}
\chapter{Catene di Markov}
Si introduce qui il concetto centrale del corso: \textbf{I processi stocastici}.
Iniziamo con il definire rigorosamente il concetto.
\begin{definition}
Sia $\{X(t),t\in T\}$ una famiglia di v.a. definite su uno spazio di probabilità $(\Omega,\mathbf{F},\mathbb{P})$ a valori in uno spazio misurabile $S$, indicizzato da un insieme ordinato $T$:
\begin{center}
    $\{X(t),t\in T\}$ è un \textbf{processo stocastico}
\end{center}
L'insieme $T$ si definisce \textbf{insieme indice} e l'insieme dei possibili valori assumibili dal processo è lo \textbf{spazio degli stati}.
\end{definition}

Classicamente un processo stocastico è indicizzato dal \textit{tempo} e dunque l'insieme indice è $\mathbb{R}_+$.
\begin{example}
Si pensi come esempi:
\begin{itemize}
    \item \textit{I passeggeri su di un bus ogni giorno ad una data ora}
    \item \textit{I passeggeri su di un bus lungo un giorno}
    \item \textit{I mm di pioggia in un anno}
\end{itemize}
\end{example}
Sono tutti processi stocastici, dove si nota che l'insieme indice e le v.a. posso prendere valori nel discreto o nel continuo. 
Un problema immediato e fondamentale nello studio dei processi stocastici è dato dallo studio della probabilità che una v.a. corrispondente ad un determinato tempo (stadio). Infatti per poterne determinare il valore abbiamo bisogno di conoscere le distribuzioni di probabilità congiunte di tutti gli stadi precedenti. Nel caso si stia parlando di v.a. \textit{indipendenti} questo problema è marginale, poichè possiamo riscrivere la congiunta come il prodotto delle singole distribuzioni. Questa condizione è molto forte però, ci viene in contro l'ipotesi di Markov.
\section{Catene di Markov}
\begin{definition}
Dato $\{X(t),t\in T\}$ processo stocastico si definisce \textbf{markoviano} se $\forall n$ vale:
\[\mathbb{P}(X_n=x_n|X_{n-1}=x_{n-1},...,X_0=x_0)=\mathbb{P}(X_n=x_n|X_{n-1}=x_{n-1})\]
Una processo di Markov è \textbf{temporalmente omogenea} se $\forall n,m\in\mathbb{N}$: \[P_{ij}(n)=P_{ij}(m)\]
Dove si denota $P_{ij}(n)=\mathbb{P}(X_n=j|X_{n-1}=i)$.
\newline
Un processo markoviano di insieme indice e spazio degli stati discreti si dice \textbf{catena di Markov}.
\end{definition}

Una catena Markoviana può essere identificata da una \textbf{matrice di transizione}: $P=(P_{ij})$
\begin{theorem}
Data $P$ una matrice di transizione e la distribuzione iniziale posso descrivere la distribuzione congiunta di ogni ordine.
\begin{proof}
Per induzione sul numero di v.a.
\begin{itemize}
    \item $n=1$: \[\mathbb{P}(X_1=x_1,X_0=x_0)=P_{01}\mathbb{P}(X_0=x_0)\] 
    \item Vale per $n$: \[\mathbb{P}(X_{n+1}=X_{n+1},X_n=x_n,...,X_0=x_0)=\]
    \[=\mathbb{P}  (X_n=x_n|X_{n-1}=x_{n-1},...,X_0=x_0)\mathbb{P}(X_{n-1}=x_{n-1},...,X_0=x_0)=\] \[=P_{n,n+1}\mathbb{P}(X_{n-1}=x_{n-1},...,X_0=x_0)\]
    Che è conosciuta per l'ipotesi induttiva.
\end{itemize}
\end{proof}
\end{theorem}
Si dice \textbf{matrice di transizione di $n$ passi} la seguente: \[P^{(n)}_{ij}=\mathbb{P}(X_{n+m}=j|X_m=i)\]
\begin{theorem}
Per una catena di Markov a tempo omogeneno vale: \[P_{ij}^{(n+m)}=\sum_kP^{(n)}_{ik}P^{(m)}_{kj}\]
\begin{proof}
\[P^{(n+m)}_{ij}=\mathbb{P}(X_{n+m}=j|X_0=i)=\sum_k\mathbb{P}(X_{n+m}=j, X_n=k|X_0=i)\]
\[\sum_k\mathbb{P}(X_{n+m}=j|X_n=k,X_0=i)\mathbb{P}(X_n=k|X_0=i)\]
Ora per l'ipotesi di Markov:
\[P^{(n+m)}_{ij}=\sum_k\mathbb{P}(X_{n+m}=j|X_n=k)\mathbb{P}(X_n=k|X_0=i)=\sum_kP^{(n)}_{ik}P^{(m)}_{kj}\]
\end{proof}
\end{theorem}
Si osserva dunque che: 
\begin{itemize}
    \item $P^{(n+m)}=P^{(n)}\cdot P^{(m)}$
    \item $P^{(n)}=P^n$
\end{itemize}
\subsection{Classificazione degli stati}
\begin{definition}
Data una catena di Markov:
\begin{itemize}
    \item Uno stato $j$ è \textbf{accessibile} da uno stato $i$ se $\exists n\in\mathbb{N}$ t.c.$P_{ij}^{(n)}>0$. Si denota con $i\longrightarrow j$.
    \item Due stati $i,j$ sono \textbf{comunicanti} se $i\longrightarrow j$ e $j\longrightarrow i$. Si denota con $i\longleftrightarrow j$.
\end{itemize}
Inoltre una catena si dice \textbf{irriducibile} se tutti gli stati sono comunicanti. Uno stato non accessibile da nessun'altro stato si dice \textbf{assorbente}.
\end{definition}
Si osserva subito che l'essere comunicanti è una \textit{relazione di equivalenza} (verificare per esercizio), e si può quindi quozientare l'insieme degli stati.

\begin{definition}
Data una catena di Markov definiamo: \[f_i=\mathbb{P}(\text{Tornare a }i | X_0=i):= \begin{cases}
=1 & \text{\textbf{Ricorrente}}
\\ <1 &\text{\textbf{Transiente}}
\end{cases}\]
\end{definition}
\begin{observation} \label{Obs211}
\begin{itemize}
    \item Se $i$ è ricorrente il numero dei ritorni a $i$ è infinito
    \item Se $i$ è transiente il processo abbandona lo stato con probabilità $1-f_i$ e $N$ il numero di passi necessari per lasciare lo stato è una v.a. geometrica con media $\frac{1}{1-f_i}$.
\end{itemize}
\end{observation}
\begin{theorem} \label{Tras_Ric}
Data una catena di Markov e uno stato $i$:
\begin{itemize}
    \item $\sum_{n=0}^{+\infty}P_{ii}^n$ diverge $\Rightarrow$ $i$ è ricorrente
    \item $\sum_{n=0}^{+\infty}P_{ii}^n$ converge $\Rightarrow$ $i$ è transiente
\end{itemize}
\begin{proof}
Sia: \[A_n= \begin{cases}
1 & X_n=i
\\ 0 & X_n\neq i
\end{cases}\]
Valutaimo ora il numero medio di volte che il processo passa da $i$, dato che vi era in $X_0$:
\[\E\left[\sum_{n=0}^{+\infty}A_n | X_0=i\right]=\sum_{n=0}^{+\infty}\E\left[A_n|X_0=i\right]=\sum_{n=0}^{+\infty}P_{ii}^n\]
Dunque, poichè uno stato è ricorrente se e solo se il numero medio di passaggi è infinito (per l'osservazione \ref{Obs211}), si ottiene la tesi.
\end{proof}
\end{theorem}
Ciò ci porta anche a concludere che:
\begin{center}
    \textit{Una catena con un numero finito di stati non può averli tutti transienti.}
\end{center}
Inoltre osserviamo che:
\begin{proposition}
La ricorrenza è una proprietà di classe.
\begin{proof}
Siano $i\longleftrightarrow j$ e $i$ ricorrente. Dunque $\exists n,m\in\mathbb{N}$ t.c. $P_{ij}^n,P_{ji}^m>0$ e $\forall k>0$ si ha:
\[P^{n+m+k}_{jj}\geq P^m_{ji}P^k_{ii}P^n_{ij}>0\]
Dunque passando alla sommatoria, poichè $\sum_{k=0}^{+\infty}P_{ii}^k$ diverge per ipotesi, si ottiene la tesi.
\end{proof}
\end{proposition}

Ci concentriamo ora su di un esempio classico e fondamentale di catena di Markov, il \textit{cammino casuale}.

\begin{example}
Definiamo la v.a. del cammino casuale come: $X_n=X_{n-1}+Y_n$ dove $Y_n$ sono IID a valori in $\{-1,1\}$  con $\mathbb{P}(Y=1)=p$. L'origine è transiente o ricorrente? Applichiamo il teorema \ref{Tras_Ric}. Osserviamo subito che:
\[P_{0,0}^{2n+1}=0\]
Infatti per poter tornare ad uno stato partendo da esso bisogna per forza compiere un numero di passi pari. Dunque:
\[\sum_{n=1}^{+\infty}P_{0,0}^n=\sum_{n=1}^{+\infty}P_{0,0}^{2n}=\sum_{n=1}^{+\infty}\binom{2n}{n}p^n(1-p)^n\]
Ricordando l' \href{https://en.wikipedia.org/wiki/Stirling%27s_approximation}{approssimazione di Stirling}  osserviamo che $\binom{2n}{n}p^n(1-p)^n$ è asintoticamente equivalente a: \[\frac{[4p(1-p)]^n}{\sqrt{\pi n}}\]
Tramite una facile analisi della convergenza della serie soprascritta:
\begin{itemize}
    \item $p=\frac{1}{2}$ l'origine è \textit{ricorrente}
    \item $p\neq\frac{1}{2}$ l'origine è \textit{transiente}
\end{itemize}
Si dimostra che questa relazione vale anche nel caso bidimensionale:
\[P_{(i,i+1),(j,j)}=P_{(i,i-1),(j,j)}=P_{(i,i),(j,j+1)}=P_{(i,i),(j,j-1)}=\frac{1}{4}\]
Si impone così la simmetria. Poichè anche nel caso bidimensionale la catena è irriducibile ci limitiamo a studiare la transienza di uno stato, l'origine:
\[\sum_{n=0}^{+\infty}P_{\textbf{0},\textbf{0}}^n=\sum_{n=0}^{+\infty}\sum_{k=0}^n\frac{(2n)!}{k!k!(n-k)!(n-k)!}\left(\frac{1}{4}\right)^{2n}\]
La seconda sommatoria e la frazione di fattoriali deriva dal fatto che, per ogni numero $k$ da $0$ a $n$, la catena per tornare nell'origine compie $k$ passi in avanti e $n-k$ in dietro e equivalentemente per su e giù (nell'altra dimensione). Ora:
\[\sum_{n=0}^{+\infty}\sum_{k=0}^n\frac{(2n)!}{k!k!(n-k)!(n-k)!}\left(\frac{1}{4}\right)^{2n}=\sum_{n=0}^{+\infty}\left(\frac{1}{4}\right)^{2n}\frac{(2n)!}{n!n!}\sum_{k=0}^n\binom{n}{k}\binom{n}{n-k}=\]
\[\sum_{n=0}^{+\infty}\left(\frac{1}{4}\right)^{2n}\binom{2n}{n}\binom{2n}{n}\]
Nuovamente tramite Stirling si determina la natura divergente della catena (mediante confronto asisntotico con una geometrica di ragione $1$). 

Si può però dimostrare che per dimensioni superiori alla seconda la catena è sempre transiente.
%In oltre anche per il cammino casuale bidimensionale vale questa relazione (dimostrato a lezione, se lo chiede all'orale la meno :)), mentre per ogni dimensione superiore \textit{non} vale.
\end{example}

Vogliamo ora calcolare $\mathbb{P}(\text{Ritorno in }0)$:
\[\mathbb{P}(\text{Ritorno in }0)=\]
\[=\mathbb{P}(\text{Ritorno in }0|X_1=1)\mathbb{P}(X_1=1)+\mathbb{P}(\text{Ritorno in }0|X_1=-1)\mathbb{P}(X_1=-1)\]
Poniamo che $p>\frac{1}{2}$ e dunque $\mathbb{P}(\text{Ritorno in }0|X_1=-1)=1$. Possiamo giustificare questo risultato osservando che, per la legge dei grandi numeri:
\[\frac{\sum_{i=1}^nY_n}{n}\xrightarrow{n\to +\infty}\E Y\neq0\]
%\[\hspace{15px} \text{valore finito }\neq0\]
E dunque la sommatoria delle $Y$ diverge: i salti in alto dominano (le $Y_n$ sono le v.a. della definizione a inizio esempio). Chiamiamo ora $\mathbb{P}(\text{Ritorno in }0|X_1=1)=\alpha$ e ricondizioniamo:
\[\alpha=\mathbb{P}(\text{Ritorno in }0|X_1=1,X_2=2)p+\mathbb{P}(\text{Ritorno in }0|X_1=1,X_2=0)(1-p)=\]\[=\alpha^2p+1-p\]
Si sottolinea il non immediato passaggio:
\[\mathbb{P}(\text{Ritorno in }0|X_1=1,X_2=2)=\mathbb{P}(\text{Ritorno in }0|X_1=1)\mathbb{P}(\text{Ritorno in}1|X_2=2)=\alpha^2\]
Ottenendo: $\alpha=\frac{1-p}{p}$. 
\newline
In definitiva $\mathbb{P}(\text{Ritorno})= \begin{cases}
2(1-p) & p>\frac{1}{2}
\\2p & p<\frac{1}{2}
\end{cases}$
\vspace{25px}


Definiamo ora alcune propietà di uno stato, e equivalentemente di una catena trattandosi di proprietà di classe. Sia:
\[N_i=min\{n\in\mathbb{N}:X_n=i\}\]
E conseguentemente:
\[m_i=\E[N_i|X_0=i]\]
\begin{definition}
Dato uno stato ricorrente $i$ si dice:
\begin{itemize}
    \item \textbf{positivo ricorrente} se $m_i < +\infty$
    \item \textbf{negativo ricorrente} se $m_i = +\infty$
\end{itemize}
\end{definition}
Si osserva che per ogni catena di Markov irriducibile a stati finiti questa è positivo ricorrente. Questo perchè:
\begin{itemize}
    \item Se fosse transiente allora dopo un numero finito di passi (il massimo tra i numeri di passaggio per ogni stato) non si troverebbe in alcuno stato, assurdo.
    \item Se fosse negativo ricorrente allora dopo un tempo finito (il massimo tra i tempi di ritorno su ogni stato) non potrebbe tornare su nessuno stato, assurdo.
\end{itemize}
Entrambe le considerazioni valgono poichè la catena è irriducibile e transienza e positivo ricorrenza sono proprietà di classe. 
\begin{definition}
Il \textbf{periodo} di uno stato $i$ è: \[d(i)=MDC\{n\geq 1 \text{  t.c.  } \mathbb{P}(X_{m+n}=i|X_m=i)>0\}\]
Uno stato è \textbf{aperiodico} se il suo periodo è 1.
\end{definition}
Evidentemente la periodicità è una proprietà di classe.
\begin{definition}
Uno stato aperiodico e positivo ricorrente si dice \textbf{ergodico}.
\end{definition}

\subsection{Distribuzioni Limite e Stazionarie}
Introduciamo il concetto di distribuzione limite. 
\begin{definition}
Una distribuzione $\pi$ per una catena di Markov è \textbf{limite} se è tale che:
\[\lim_{n\to+\infty}P_{ij}^n=\pi_j \hspace{7px} \forall i\]
\end{definition}
\vspace{5px}
Così possiamo enunciare il seguente teorema, tratteggiandone un'idea della dimostrazione.
\begin{theorem} \label{Teor_erg}
Data una catena di markov irriducibile e ergodica allora:
\begin{itemize}
    \item $\lim_{n\to+\infty}P_{ij}^n=\pi_j$ esiste finito (esiste una distribuzione limite)
    \item $\sum_j\pi_j=1$ e $\pi_j$ \textbf{non} dipende da $n$
    \item $\{\pi_j\}$ sono l'unica soluzione non negativa del sistema: \begin{equation} \label{Sist_erg}
    \begin{cases}
    \pi_j=\sum_i\pi_iP_{ij}
    \\ \sum_j\pi_j=1
    \end{cases}
    \end{equation}
\end{itemize}
\begin{proof}
Ammettendo l'esistenza di $\{\pi_j\}$ che soddisfano la prima condizione e verifico che siano soluzione di \ref{Sist_erg}:
\[\mathbb{P}(X_{n+1}=j)=\sum_i\mathbb{P}(X_{n+1}=j|X_n=i)\mathbb{P}(X_n=i)=\sum_iP_{ij}\mathbb{P}(X_n=i)\]
Passando al limite, ammettendo che si possa invertire la sommatoria con il limite, si ottiene la tesi.
\end{proof}
\end{theorem}
%Per aiutare la visualizzazione dei $\pi_i$ li si può immaginare come la proporzione di tempo passato dalla catena sullo stato $i$-esimo per $n\longrightarrow+\infty$.

Si faccia particolare attenzione al fatto che il sistema \ref{Sist_erg}, intendendo $\overline{\pi}_j$ come vettore, si riscrive come:
\[\overline{\pi}_j=\overline{\pi}_i\cdot P\]
E si deve dunqe rispettare l'ordine di moltiplicazione dei fattori della matrice di transizione con il vettore: il prodotto matriciale \textit{non} commuta.

Passiamo ora a definire un'altra tipologia di distribuzione.
%\begin{definition}
%Una distribuzione $\pi=(\pi_i)$ di una catena di Markov con matrice di transizione $P$ è \textbf{stazionaria} se:
%\[\pi=\pi P\]
%\end{definition}
%Ciò che afferma la definizione è che la distribuzione rimane costante nel tempo, non dipende dall'indice scelto.
%\vspace{5px}
%Vale a dire che la distribuzione $\overline{\pi}$ (d'ora in avanti sarà solo $\pi$) sarà tale che:
%\[\mathbb{P}(X_0=i)=\pi_i \Rightarrow \mathbb{P}(X_n=i)=\pi_i \hspace{5px} \forall n\in\mathbb{N}\]

\begin{definition}
Una distribuzione $\pi=(\pi_i)$ di una catena di Markov è \textbf{stazionaria} se
:
\[\mathbb{P}(X_0=j)=\pi_j \Rightarrow \mathbb{P}(X_1=j)=\pi_j \]
\end{definition}
Ciò che afferma la definizione è che la distribuzione rimane costante nel tempo. Si verifica infatti (banale dimostrazione per induzione) che una distribuzione è stazionaria se e solo se vale:
\[\pi=\pi \cdot P\]

\begin{proposition}
Una distribuzione limite è stazionaria.
\begin{proof}
Ammesso che la distribuzione limite dipenda dalla distribuzione iniziale, \textit{i.e.}:
\[\pi_j^{(i)}=\lim_{n\to+\infty}\mathbb{P}(X_n=j|X_0=i)\]
Dunque passiamo a calcolare:
\[\pi_j^{(i)}=\lim_{n\to+\infty}\mathbb{P}(X_n=j|X_0=i)=\]
\[\lim_{n\to+\infty}\sum_{k}\mathbb{P}(X_n=j|X_{n-1}=k)\mathbb{P}(X_{n-1}=k|X_0=i)=\sum_{k}\lim_{n\to+\infty}P_{kl}\cdot\mathbb{P}(X_{n-1}=k|X_0=i)=\]
\[\sum_k\pi_k^{(i)}P_{kl}\]
Dunque la distribuzione è stazionaria.
\end{proof}
\end{proposition}

Si può subito osservare però che non vale il contrario: 
\begin{center}
    Stazionarietà non implica ammetterre limite
\end{center}
Per verificare ciò basta prendere una qualsiasi catena \textit{non} irriducibile che ammette distribuzione staionaria.
Togliendo l'rriducibilità non vale più il teorema \ref{Teor_erg}, ma anche privandosi dell'\textit{aperiodicità} sorgono dei problemi. In questi casi si può intendere $\pi$ come la proporzione di tempo passato sul quello stato (una distribuzione stazionaria).

Riscrivendo la \ref{Sist_erg} si ottiene l'\textbf{equazione di bilanciamento}:
\begin{equation} \label{Eq_bil}
    (1-P_{jj})\pi_j=\sum_{i\neq j}P_{ij}\pi_i
\end{equation}
Ricordando invece $m_j$ come il valore atteso delle transizioni necessarie alla catena per tornare a $j$ essendo partita da $j$:
\begin{equation}
    m_j=\E[N_j|X_0=j]
\end{equation}
Dove $N_j=min\{n>0|X_n=j\}$.

Possiamo enunciare il \textbf{teorema ergodico}. %(Possibile domanda d'esame)
\begin{theorem}
Data una catena si Markov irriducibile, la media campionaria del numero di visite allo stato $j$ converge q.c. a $\frac{1}{m_j}$. In formule:
\[\lim_{n\to+\infty}\frac{1}{n}\sum_{k=0}^n\mathbb{I}_{\{X_k=j\}}=\frac{1}{m_j}\]
Se la catena è positivo ricorrente:
\[\lim_{n\to+\infty}\frac{1}{n}\sum_{k=0}^n\mathbb{I}_{\{X_k=j\}}=\pi_j \hspace{15px} \text{Converge q.c.}\]
Dove $\pi$ è la distribuzione stazionaria. Se la catena è ergodica:
\[\pi_j=\frac{1}{m_j}\]
\end{theorem}
\newpage

\subsection{Esempi di catene di Markov}

\subsubsection{Processi di diramazione}
I processi di diramazione sono un esempio di catene di Markov con varie applicazioni i numerosi settori della ricerca scientifica. L'idea sostanziale è valutare l'evoluzione di una popolazione i cui membri producono prole.

In una popolazione ogni individuo produce $j$ individui con probabilità $P_j$, indipendentemente dagli altri e con $P_j<1$. Si definisce la catena di Markov:
\begin{center}
    $\{X_n\}_{n\geq0}$ che conta il numero di individui della popolazione al tempo $n$
\end{center}
$X_i$ è detta $i$-esima \textbf{generazione}. 

Notiamo ad esempio che lo stato $0$ è assorbente e che se $P_0>0$ allora tutti gli stati sono transienti: in particolare la catena o va a $0$ o diverge.

Sia invece:
\[\mu=\sum_{j=0}^{+\infty}jP_j\]
Il numero medio di individui generati da un individuo. Osserviamo invece che possiamo riscrivere l'$n$-esima generazione:
\[X_n=\sum_{i=0}^{X_{n-1}}Z_i\]
Dove $Z_i$ conta il numero di individui generati dall'$i$-esimo individuo. Dunque:
\[\mathbb{E}X_n=\mathbb{E}\left[\mathbb{E}\left(X_n|X_{n-1}\right)\right]=\mathbb{E}\left[\mathbb{E}\left(\sum_{i=0}^{X_{n-1}}Z_i|X_{n-1}\right)\right]=\mathbb{E}(X_{n-1}\mathbb{E}Z_i)=\mu\mathbb{E}X_{n-1}\]

Iterando e ponendo $X_0=1$ ottengo $\mathbb{E}X_n=\mu^n$.

Con ragionamenti analogi (relativi alla doppia attesa e tramite il teorema \ref{Varr}) si calcola $\mathbb{V}ar X_n$.

Consideriamo invece la probabilità limite, in particolare $\pi_0$ cioè la probabilità che all'infinito la popolazione si estingua (sempre posto che si parta da un solo individuo). 
\[\pi_0=\lim_{n\to+\infty}\mathbb{P}(X_n=0|X_0=1)\]
Si nota che se $\mu<1$ allora $\pi_0=1$, infatti:
\[\mu^n=\sum_{j=1}^{+\infty}j\mathbb{P}(X_n=j)\geq \sum_{j=1}^{+\infty}\mathbb{P}(X_n=j)=\mathbb{P}(X_n\geq 1)\]
Dunque:
\[\mathbb{P}(X_n=0)=1-\mathbb{P}(X_n\geq1)\geq1-\mu^n\xrightarrow{n\to+\infty}1\]
Dunque per il teorema del confronto (una probabilità è sempre minore o uguale a 1) si ottiene:
\[\mu<1\Rightarrow\pi_0=1\]
Si dimostra anche che $\mu=1$ implica $\pi_0=1$.

Più in generale vale:
\[\pi_0=\sum_{j=1}^{+\infty}\pi_0^jP_j\]
Dove $\pi_0^j=\mathbb{P}(\text{Popolazione va a }0|X_0=j)$. Per provare quest'uguaglianza basta condizionare rispetto a $P_j$.
\vspace{15px}

\subsubsection{Rovina del giocatore} \label{Rov_gioc}
Giocatore fa giocate successive con probabilità $p$ di vincere un euro e $1-p$ di perderlo. Le giocate sono indipendenti. Qual è la probabilità di raggiungere un capitale $N$ prima di andare in rovina?
\newline
\textbf{Soluzione.}
\newline
Si definisce $X_n=$"fortuna del giocatore  al tempo $n$". Dunque $\{X_n ; n\in\mathbb{N}\}$ è una catena di Markov, un cammino casuale con due barriere. La sua matrice di transizione è una matrice infinita della forma:
\[P = \begin{bmatrix}
1& 0 & 0 & 0
\\q & 0 & p & 0 
\\0 & q & 0 & p 
\\0 & 0 & q & 0 
\end{bmatrix}
\quad\]
La catena è dotata di tre classi $\{0\},\{1,2,\dots,N-1\},\{N\}$, dove la prima e la terza sono ricorrenti mentre la seconda è transiente, dunque dopo un tempo finito il giocatore ho vince (finisce in $N$) o perde tutto (finisce in $0$).

Definita $P_i = $ "probabilità di raggiungere $N$ prima di $0$ partendo da $i$":
\[P_i=(P_i|X_1=i+1)P_{i,i+1}+(P_i|X_1=i-1)P_{i,i-1}=P_{i+1}p+P_{i-1}q\]
Usando che $p+q=1$ e dunque $P_i=P_ip+P_iq$ si ottiene:
\[P_{i+1}-P_i=\frac{q}{p}(P_i-P_{i-1})\]
Dunque:
\[P_i-P_{i-1}=\frac{q}{p}(P_{i-1}-P_{i-2})=\dots=\left(\frac{q}{p}\right)^{i-1}P_1\]
Cioè:
\[\cdot P_i=\sum_{k=0}^{i-1} \left (\frac{q}{p}\right)^k P_1\]
\[\cdot P_i= \begin{cases}
\frac{1-\left(\frac{q}{p}\right)^i}{1-\left(\frac{q}{p}\right)}P_1 & p\neq \frac{1}{2}
\\ iP_1 & p=\frac{1}{2}
\end{cases}\]
Usando $P_N=1$:
\[P_i= \begin{cases}
\frac{1-\left(\frac{q}{p}\right)^i}{1-\left(\frac{q}{p}\right)^N} & p\neq \frac{1}{2}
\\ \frac{i}{N} & p=\frac{1}{2}
\end{cases}\]
Dunque per $N\rightarrow+\infty$ la $P_i$ converge solo per $p>\frac{1}{2}$, infatti se $p\leq \frac{1}{2}$ le situazioni possono essere:
\begin{itemize}
    \item $p=\frac{1}{2}$: Dunque $P_i=\frac{i}{N}\xrightarrow{N\to+\infty}0$
    \item $p<\frac{1}{2}$: Dunque $p<q$ perciò $\frac{q}{p}>1$ e quindi $1-\left(\frac{q}{p}\right)^N\xrightarrow{N\to+\infty}+\infty$
\end{itemize}

Il giocatore dunque potrà aumentare il suo capitale indefinitamente solo per una probabilità di vincita maggiore di $1/2$.
\chapter{Processi di Poisson}
\section{processi di conteggio e di Poisson}
Introduciamo ora una nuova famiglia di processi, i \textbf{processi di conteggio}. I processi di conteggio, come suggerisce il nome, contano il numero di eventi che si verificano in un dato intervallo di tempo. Si passa dunque al continuo per l'insieme indice.
\begin{definition}
Il processo stocastico $\{N(t)\}_{t\geq 0}$ è di \textbf{conteggio} se:
\begin{itemize}
    \item $N(t)\geq 0$
    \item $N(t)\in\mathbb{N}$
    \item $N(t)\leq N(s)$ per $t\leq s$
    \item La quantità $N(t)-N(s)$ detta \textbf{incremento} conta il numero di eventi nell'intervallo di tempo $(s,t]$
\end{itemize}
Un processo è a \textbf{incrementi indipendenti} se il numero di eventi che si verificano su intervalli disgiunti sono variabili aleatorie indipendenti. Un processo è a \textbf{incrementi stazionari} se il numero di eventi che avvengono in un lasso di tempo dipende solo dalla lunghezza di esso, e non dal momento iniziale: \[N(t+s)-N(s)\sim N(t)-N(0)\]
$\forall s,t\in\mathbb{R}_+$.
\end{definition}
Così definiti i processi di conteggio possiamo definire quelli di Poisson.
\begin{definition}[Poisson 1]
Un processo di conteggio $\{N(t)\}_{t\geq0}$ è di \textbf{Poisson} se:
\begin{itemize}
    \item $N(0)=0$
    \item A incrementi indipendenti
    \item A incrementi stazionari (\textit{condizione sovrabbondante. Si può derivare dalle altre.})
    \item $\mathbb{P}(N(t)-N(s)=k)=\frac{(\lambda(t-s))^k}{k!}e^{-\lambda(t-s)}$
\end{itemize}
\end{definition}

Il processo di Poisson può essere definito in altri modi equivalenti, facendo uso degli incrementi possibili in un lasso di tempo infinitesimo. In particolare in $[t,t+h)$ se $h$ è infinitesimo potrà verificarsi al più un evento, quindi:
\[\mathbb{P}(N(t+h)-N(t)=1)=\lambda he^{-\lambda h}=\lambda h(1-\lambda h + o(h))=\lambda h + o(h)\]
\[\mathbb{P}(N(t+h)-N(t)=0)=1-\lambda h+o(h)\]
Dunque si può dare l'equivalente definizione:
\begin{definition}[Poisson 2]
Un processo di conteggio $\{N(t)\}_{t\geq0}$ è di \newline \textbf{Poisson} se:
\begin{itemize}
    \item $N(0)=0$
    \item A incrementi indipendenti e stazionari
    \item $\mathbb{P}(N(h)=1)=\lambda h+o(h)$
    \item $\mathbb{P}(N(h)\geq2)=o(h)$
\end{itemize}
\end{definition}
Si mostra che le due definizioni sono equivalenti:
\begin{itemize}
    \item $1\Rightarrow 2$: è gia stato dimostrato sopra.
    \item $2\Rightarrow 1$: Sia $u\geq0$ fissato e considero la trasformata di Laplace:
    \[g(u)=\mathbb{E}\left[e^{-uN(t)}\right]\]
    Calcolo:
    \[g(t+h)=\mathbb{E}\left[e^{-uN(t+h)}\right]=\mathbb{E}\left[e^{-u(N(t+h)-N(t)+N(t))}\right]\]
    Per indipendenza e stazionarietà del processo:
    \[g(t+h)=\mathbb{E}\left[e^{-uN(h)}\right]\mathbb{E}\left[e^{-uN(t)}\right]=g(t)\mathbb{E}\left[e^{-uN(h)}\right]\]
    Dunque calcolo:
    \[\mathbb{E}\left[e^{-uN(h)}\right]=e^{-u\cdot0}(1-\lambda h +o(h))+e^{-u\cdot 1}(\lambda h+o(h))+o(h)=\]
    \[1-\lambda h+e^{-u}\lambda h+ o(h)\]
    Ottenendo:
    \[g(t+h)-g(t)=g(t)\big(h\lambda(e^{-u}-1)+o(h)\big)\]
    E dunque:
    \[g'(t)=\lim_{h\to0}\frac{g(t+h)-g(t)}{h}=g(t)\lambda(e^{-u}-1)\]
    Ottenendo infine:
    \[\int_0^t\frac{dg}{g}=\int_0^t\lambda(e^{-u}-1) \,dt \Rightarrow g(t)=g(0)e^{\lambda t(e^{-u}-1)}\]
    Che è la trasformata di Laplace di una Poisson di parametro $\lambda t$.
\end{itemize}
\subsection{Intertempi}
\vspace{5px} 
Un importante fattore caratterizzante per un processo di Poisson sono gli \textbf{intertempi} tra eventi, cioè:
\begin{center}
$T_i$ = Tempo attesa dall'evento $(i-1)$-esimo all'$i$-esimo
\end{center}
Studiamone la distribuzione, osservando che $\{T_1>t\}=\{N(t)=0\}$ come eventi:
\[\mathbb{P}(T_1>t)=\mathbb{P}(N(t)=0)=e^{-\lambda t}\]
Equivalentemente:
%osservando che $\{T_2>t|T_1=s\}=\{N(t+s)-N(s)=0\}=\{N(t)=0\}$, si ottiene:
\[\mathbb{P}(T_2>t|T_1=s)=\frac{\mathbb{P}(T_2>t,T_1=s)}{\mathbb{P}(T_1=s)}=\frac{\mathbb{P}(N(t+s)-N(s)=0,N(s)=1)}{\mathbb{P}(N(s)=1)}\]
Poichè gli incrementi sono indipendenti si riduce:
\[\mathbb{P}(N(t+s)-N(s)=0)=e^{-\lambda t}\]
%Ora si usa questo risultato in:
%\[\mathbb{P}(T_2>t)=\mathbb{E}(\mathbb{P}(T_2>t|T_1))=\int_0^tse^{-\lambda t} \,ds\]
%\[(N(t)=0)=e^{-\lambda t}\]
Dunque, procedendo iterativamente (osservando che $\mathbb{P}(T_2>t|T_1=s)$ \textit{non} dipende da $s$) gli intertempi sono v.a. IID di distribuzione $Exp(\lambda)$. Grazie a questa caratterizzazione si può dare una terza definizione equivalente di Processo di Poisson.
\begin{definition}[Poisson 3]
Una processo di conteggio con intertempi IID come esponenziali di parametro $\lambda$ è un processo di Poisson di parametro $\lambda$.
\end{definition}

\begin{observation}
Il processo di Poisson è \textit{markoviano} con matrice di transizione:
\[P=\begin{bmatrix}
0 & 1 & 0 & \dots \\
0 & 0 & 1 & 0 \\
0 & 0 & 0 & 1
\end{bmatrix}\]
Questo è ovvio poichè il processo può solo salire, e questo evviene con probabilità $1$. In quanto tempo venga invece è caratterizzato dal parametro della Poisson in considerazione.
\end{observation}

\subsection{Ulteriori proprietà del processo di Poisson}
Sia $\{N(t)\}$ un processo di Poisson. Ogni evento è suddiviso tra tipo I e tipo II, definiamo $N_1(t)$ e $N_2(t)$ i processi di conteggio che contano rispettivamente i tipi I e II. Si verifica che $\{N_1(t)\},\{N_2(t)\}$ sono due processi di Poisson di parametri $\lambda p$ e $\lambda(1-p)$:
\begin{itemize}
    \item $N_1(0)=0$ poichè $N(0)=0$ 
    \item $N_1$ eredita incrementi indipendenti e stazionari
    \item \begin{itemize}
        \item \[\mathbb{P}(N_1(h)=1)=\]
        \[\mathbb{P}(N_1(h)=1|N(h)=1)\mathbb{P}(N(h)=1)+\mathbb{P}(N_1(h)=1|N(h)\geq2)\mathbb{P}(N(h)\geq2)\]
        \[p(\lambda h+o(h))+o(h)=\lambda ph+o(h)\]
        \item 
        \[\mathbb{P}(N_1(h)\geq2)\leq\mathbb{P}(N_1\geq2)=o(h)\]
    \end{itemize}
\end{itemize}
Equivalentemente si dimostra per $N_2(t)$.\vspace{10px}

Consideriamo due processi di Poisson indipendenti e $S_n^1$ e $S_m^2$ i tempi dell'$n$-esimo e $m$-esimo evento rispettivamente per il processo $1$ e per il $2$. Vogliamo studiare $\mathbb{P}(S_n^1<S_m^2)$.
\begin{itemize}
    \item $m=n=1$: \[\mathbb{P}(S_n^1<S_m^2)=\mathbb{P}(T_1<T_2)=\frac{\lambda_1}{\lambda_1+\lambda_2}\]
    \item $n=2,m=1$: \[\mathbb{P}(S_n^1<S_m^2)=\mathbb{P}(S_2^1<S_1^2|S_1^1<S_1^2)\mathbb{P}(S_1^1<S_1^2)=\]
    \[\mathbb{P}(S_1^1<S_1^2)\mathbb{P}(S_1^1<S_1^2)=\left(\frac{\lambda_1}{\lambda_1+\lambda_2}\right)^2\]
\end{itemize}
%Il ragionamento applicato sfrutta la mancanza di memoria. 
Ogni volta che si verifica uno dei due processi la probabilità che il prossimo a verificarsi di $1$ è $p_1=\frac{\lambda_1}{\lambda_1+\lambda_2}$ mentre che sia $2$ è $p_2=\frac{\lambda_2}{\lambda_1+\lambda_2}$. Così ragionando si ottiene:
\[\mathbb{P}(S_n^1<S_m^2)=\sum_{k=n}^{n+m-1}\binom{n+m-1}{k}\left(\frac{\lambda_1}{\lambda_1+\lambda_2}\right)^k\left(\frac{\lambda_2}{\lambda_1+\lambda_2}\right)^{n+m-1-k}\]

\subsection{Distribuzione condizionale dei tempi di arrivo}

Viene chiesto di determinare la distribuzione del tempo necessario affinchè un evento in un processo di Poisson si verifichi, sapendo che un evento si è verificato in $[0,t]$. Poichè il conteggio di Poisson e a incrementi stazionari e indipendenti il tempo di realizzazione deve essere uniformemente distribuito su $[0,t]$:
\[\mathbb{P}(T_1<s|N(t)=1)=\frac{\mathbb{P}(T_1<s,N(t)=1)}{\mathbb{P}(N(t)=1)}=\frac{\mathbb{P}(N_{[0,s)}=1,N_{[s,t]}=0)}{\mathbb{P}(N(t)=1)}=\]
\[\frac{\mathbb{P}(N_{[0,s)}=1)\mathbb{P}(N_{[s,t]}=0)}{\mathbb{P}(N(t)=1)}=\frac{\lambda se^{-\lambda s}}{\lambda te^{-\lambda t}}e^{-\lambda(t-s)}=\frac{s}{t}\]

Introduciamo il concetto di statistiche d'ordine.
\begin{definition}
Siano $\{Y_n\}$ v.a. IID si dice che $Y_{(1)},...,Y_{(n)}$ sono le \textbf{statistiche d'ordine} corrispondenti se a $Y_{(k)}$ corrisponde il $k$-esimo valore più piccolo tra le v.a.
\end{definition}

\begin{proposition}
Date $\{Y_n\}$ v.a.  assolutamente continue IID di densità $f$, allora la densità della statistica d'ordine è:
\[\overline{f}(y_1,...,y_2)=n!\prod_{i=1}^nf(y_i)\]
\begin{proof}
$Y_{(1)},...,Y_{(n)}$ è uguale a $y_{1},...,y_{n}$ se $\{Y_n\}$ è uguale a una delle $n!$ permutazioni di $y_{1},...,y_{n}$. Ora poichè $f(y_1,...,y_2)=\prod_{i=1}^nf(y_i)$ allora si ottiene la tesi.
\end{proof}
\end{proposition}

Ora, se $Y_n$ sono uniformemente distribuite su $[0,t]$ allora: \[\overline{f}(y_1,...,y_n)=\frac{n!}{t^n}\]
Ciò ci porta a:
\begin{theorem}
I tempi di arrivo dato $N(t)=n$ hanno la stessa distribuzione della statistica ordinata corrispondente a $n$ v.a. IID uniformemente distribuite su $[0,t]$.
\begin{proof}
Si osserva che per $0\geq s_1\geq...\geq s_n$ vale l'uguaglianza di eventi:
\[\left\{S_1=s_1,...,S_n=s_n,T(t)=n\right\}=\left\{T_1=s_1,...,T_n=s_n-s_{n-1},T_{n+1}>t-s_n\right\}\]
Dunque:
\[f(s_1,...,s_n|n)=\frac{f(s_1,...,s_n,n)}{\mathbb{P}(T(t)=n)}=\frac{\lambda e^{-\lambda s_1}\dots\lambda e^{-\lambda (s_n-s_{n-1})}e^{-\lambda (t-s_n)}}{\frac{\left(\lambda t\right)^n}{n!}e^{-\lambda t}}=\frac{n!}{t^n}\]
\end{proof}
\end{theorem}

\subsection{Generalizzazioni di processi di Poisson}
\subsubsection{Porcesso di Poisson nonomogeneo}
\begin{definition}[Poisson 2]
Un processo di conteggio $\{N(t)\}_{t\geq0}$ è di \newline \textbf{Poisson nonomogeneo} con \textbf{funzione d'intensità} $\lambda(t)$ se:
\begin{itemize}
    \item $N(0)=0$
    \item A incrementi indipendenti 
    \item $\mathbb{P}(N(t+h)-N(t)=1)=\lambda(t) h+o(h)$
    \item $\mathbb{P}(N(t+h)-N(t)\geq2)=o(h)$
\end{itemize}
Si definisce inoltre la \textit{funzione del valore atteso di una Poisson nonomogenea}:
\[m(t)=\int_0^t\lambda(y) \,dy\]
\end{definition}

Si dimostra che:
\[\mathbb{P}(N(t+s)-N(t)=n)=e^{m(t+s)-m(t)}\frac{[m(t+s)-m(t)]^n}{n!}\]

\subsubsection{Porcesso di Poisson composto}

Un processo di Poisson si dice \textbf{composto} se è rappresentabile come:
\[X=\sum_{i=1}^{N(t)}Y_i\]
Dove $Y_i$ sono v.a. IID e $N(t)$ un processo di Poisson indipendente dalle v.a.
\chapter{Catene di Markov a tempo continuo}
\section{Definizioni}
\begin{definition}
Un processo stocastico $\{X(t)\}_{t\geq0}$ a valori sugli interi non negativi è una \textbf{catena di markov a tempo continuo} se:
\[\mathbb{P}(X(t)=j|X(s)=i,X(u)=x(u),0\leq u<s)=\mathbb{P}(X(t)=j|X(s)=i)\]
$\forall s,t\geq0$ reali e $\forall i,j,x(u)\in\mathbb{N}^*$.
\newline
Se $\mathbb{P}(X(t+s)=j|X(s)=i)$ è indipendente da $s$ allora la catena è a \textbf{incrementi omogenei}. 
\end{definition}

Si osserva che la condizione di Markovianità implica che,
%\[\mathbb{P}(X(t+s)=i|X(s)=i)=\mathbb{P}(X(t)=i|X(0)=i)\]
 considerando $T_i$ il tempo durante il quale il processo rimane nello stato $i$ si ottiene:
\[\mathbb{P}(T_i>t+s|T_i>s)=\mathbb{P}(T_i>t)\]
Dunque le $T_i$ sono prive di memoria e quindi esponenziali.

Una definizio equivalente è quindi possibile.

\begin{definition}
Una catena di Markov è a \textbf{tempo continuo} se:
\begin{itemize}
    \item Il tempo trascorso dal processo su uno stato $i$ è un'esponenziale di media $\frac{1}{v_i}$
    \item Quando il processo lascia uno stato $i$ raggiunge uno stato $j$ con probabilità $P_{ij}$ tale che: $\begin{cases}
    \sum_{j}P_{ij}=1 \\
    P_{ij}\geq0 & \forall i,j
    \end{cases}$
\end{itemize}
\end{definition}

Deriva immediatamente che gli intertempi devono essere \textbf{indipendenti}. Così non fosse le informazioni su quanto tempo la catena passa su uno stato influirebbero sulla probabilità di passaggio al successio, in \textit{contrasto} con la markovianità della catena.

\section{Processi di nascita-morte}
Forse il più importante esempio e applicazione di catene di Markov a tempo continuo sono i modelli di nascita morte.
\begin{itemize}
    \item Sia una catena di Markov con stati il numero di individui in una popolazione (spazio degli stati $\mathbb{N}^*$).
    \item Allo stato $n$ i tempi di nascita e morte sono esponenziali di parametro $\lambda_n,\mu_n$
\end{itemize}
Dunque il tempo medio prima di una nascita è $\frac{1}{\lambda_n}$ e quello prima di una morte è $\frac{1}{\mu_n}$. Da questo possiamo ricavare il tempo medio di attesa in uno stato $n$, questo sarà il valore atteso del minimo degli esponenziali di parametro $\lambda_n$ e $\mu_n$, cioè:
\[\text{"Attesa media in n"}=\frac{1}{v_n}=\frac{1}{\lambda_n+\mu_n}\]

\begin{example}
Sia un processo nascita-morte di parametri:$\begin{cases}
\lambda_n=n\lambda+\theta \\
\mu_n=n\mu
\end{cases}$.
Anche detto a crescita lineare con immigrazione e morte lineare.
\newline
Si vuole studiare $M(t)=\mathbb{E}X_t$, dove $X_t$ conta gli individui al tempo $t$. Si procede sfruttando la markovianità per determinare una equazione differenziale per lo studio di $M$.

\[M'=lim_{h\to0}\frac{M(t+h)-m(t)}{h}\]
Ora valuto $M(t+h)$ tramite la doppia attesa:
\[\mathbb{E}X(t+h)=\mathbb{E}[\mathbb{E}(X(t+h)|X(t))]\]
nel tempo infinitesimo $h$ la catena può scendere di 1 rimanere invariata o salire di 1. Tutto ciò viene fatto con probabilità:
\[\mathbb{P}\Big(X(t+h)=X(t)+1\Big)=\mathbb{P}(T_{X(t)}^{\uparrow}<h)=1-e^{-\lambda_{X(t)}h}=\]
\[1-(1-\lambda_{X(t)}h-o(h))=(\lambda X(t)+\theta)h+o(h)\]
Dove $T_{X(t)}^{\uparrow}$ è il tempo per fare una transizione in alto da $X(t)$ individui. Con ragionamenti analoghi si ottiene:
\[X(t+h)=\begin{cases}
X(t)+1 & \text{con probabilità}: (\lambda X(t)+\theta)h+o(h) \\
X(t) & \text{con probabilità}: 1-[(\lambda X(t)+\theta+\mu X(t))h + o(h)] \\
X(t)-1 & \text{con probabilità}: \mu X(t)h + o(h)
\end{cases}\]
Dunque ottengo:
\[M(t+h)=M(t)+(\lambda-\mu)M(t)h+\theta h+o(h)\]
Quindi raccogliendo e passando al limite ottengo:
\[M'(t)=(\lambda-\mu)M(t)+\theta\]
E risolvendo l'equazione differenziale si trova la distribuzione di $M(t)$:
\[M(t)=\frac{\theta}{\lambda-\mu}(e^{(\lambda-\mu)t}-1)+X(0)e^{(\lambda-\mu)t}\]
\end{example}

\subsection{Funzione di transizione di probabilità}

Si vuole ora definire ora un'equivalente delle equazioni di Chapman-Kolmogorov per le catene di Markov a tempo continuo. Definiamo:
\[\mathbb{P}(X(t+s)=j|X(s)=i)=P_{ij}(t)\]
Come \textbf{funzione di transizione di probabilità}. Invece siano:
$\begin{cases}
v_i : \text{Tasso di abbandono di i} \\
q_{ij} : \text{Tasso con cui da i si va in j}
\end{cases}$

E osserviamo $q_{ij}=v_iP_{ij}$ e che:
\[\mathbb{P}(X(t+h)=j|X(t)=i)=\mathbb{P}(T(i)^j\leq h)=1-e^{-q_{ij}h}=q_{ij}h+o(h)\]
Dove $T(i)^j$ è il tempo per passare da $i$ a $j$, e:
\[P_{ij}=\frac{q_{ij}}{\sum_jq_{ij}}\]
Tramite i due seguenti lemmi arriviamo alla formulazione dell'equazione differenziale che caratterizzerà un equazione di Markov a tempi continui.
\begin{lemma}
\[\cdot \lim_{h\to0}\frac{1-P_{ii}(h)}{h}=v_i\]
\[\cdot \lim_{h\to0}\frac{P_{ij}(h)}{h}=q_{ij}\]
\vspace{5px}
\begin{proof}
$\cdot 1-P_{ii}(h)=v_ih+o(h)$
Diviso $h$ al limite vale $v_i$.

$\cdot P_{ij}(h)=q_{ij}h+o(h)$
Diviso $h$ al limite vale $q_{ij}$.
\end{proof}
\end{lemma}

\begin{lemma}
$\forall s,t>0$ vale: 
\[P_{ij}(t+s)=\sum_{k=0}^{+\infty}P_{ik}(t)P_{kj}(s)\]
\begin{proof}
Si condiziona.
\end{proof}
\end{lemma}

Dunque si può finalmente enunciare il seguente teorema.
\begin{theorem}[Kolmogorov Backward]
$\forall i,j$ e $t\geq0$ vale:
\[P^{'}_{ij}(t)=\sum_{k\neq i}\left(q_{ik}P_{kj}(t)\right)-v_iP_{ij}(t)\]
\begin{proof}
Inizio a valutare grazie al secondo lemma:
\[P_{ij}(t+h)-P_{ij}(t)=\sum_kP_{ik}(h)P_{kj}(t)-P_{ij}(t)=\sum_{k\neq i}P_{ik}(h)P_{kj}(t)-(1-P_{ii}(h))P_{ij}(t)\]
Dunque dividendo per $h$ e passando al limite, tramite il primo lemma ottengo la tesi.
\end{proof}
\end{theorem}

Un simile risultato si ottiene nel seguente teorema.

\begin{theorem}[Kolmogorov Forward]
$\forall i,j$ e $t\geq0$ vale:
\[P^{'}_{ij}(t)=\sum_{k\neq j}\left(q_{kj}P_{ik}(t)\right)-v_jP_{ij}(t)\]
\end{theorem}
\begin{proof}
Si dimostra equivalentemente al teorema precedente, condizionando su $(0,t)\cup(t,t+h)$.
\end{proof}

\section{Distribuzioni limite}

In analogia con il caso discreto definiamo:
\[P_j=\lim_{t\to+\infty}P_{ij}(t)\]

Dunque, ammessa l'esistenza di questo limite ricaviamo le \textbf{equazioni di bilanciamento} in questo modo:
\[\lim_{t\to+\infty}P^{'}_{ij}(t)=\lim_{t\to+\infty}\left[\sum_{k\neq j}q_{kj}P_{ik}(t)-v_{j}P_{ij}(t)\right]=\sum_{k\neq j}q_{kj}P_k-v_{j}P_j\]

Poichè $P_{ij}(t)$ è limitata in $[0,1]$ al limite la derivata deve essere nulla, se così non fosse al limite la funzione uscirebbe dall'intervallo. Dunque, imponendo $\lim_{t\to+\infty}P_{ij}^{'}(t)=0$ si ottiene la vera e propria equazione di bilanciamento:
%Ponendo $P^{'}_{ij}(t)=0$ come condizione di stazionarietà della distribuzione limite si ottiene la vera e propria equazione di bilanciamento:
\begin{equation}\label{Eq_bil}
\sum_{k\neq j}q_{kj}P_k=v_jP_j    
\end{equation}

\newpage
Dove:
\begin{itemize}
    \item $\sum_{k\neq j}q_{kj}P_k$ è il tasso di entrata in $j$ (da ogni stato $k$)
    \item $v_jP_j$ è il tasso di uscita da $j$
\end{itemize}
Dunque l'equzione di bilanciamento, come suggerisce il nome, impone che i tassi di entrata ed uscita da uno stato per una distribuzione limite siano uguali. Questo è in completo accordo con il caso discreto, e riafferma la natura stazionaria di una distribuzione limite.

Nella derivazione della \ref{Eq_bil} abbiamo dato per vera l'esistenza della distribuzione limite, della quale possiamo però fornire delle condizioni sufficienti:
\begin{itemize}
    \item La catena è irriducibile
    \item La catena è positivo ricorrente
\end{itemize}

Si presenta ora un esempio pregno di significato, la cui risoluzione può fare da modello per altri esercizi.
\begin{example}
Sia processo di nascita morte: $\begin{cases}
\lambda_n=1 & n=0,1,\dots \\
\mu_n=n & n=1,2,\dots
\end{cases}$


Ora per calcolare il tempo medio di passaggio definendo $T_i$ tempo per andare da $i$ a $i+1$, si ha banalmente per $n=0$:
\[\mathbb{E}T_0=\frac{1}{\lambda_0}=1\]
Mentre per $n\neq1$ definiamo $I_i=\begin{cases}
1 & i\longrightarrow i+1 \\
0 & i\longrightarrow i-1
\end{cases}$, dunque:
\[\begin{cases}
\mathbb{P}(I_i=1)=\mathbb{P}\left(min\{\lambda_i,\mu_i\}=\lambda_i\right)=\frac{\lambda_i}{\lambda_i+\mu_i} \\
\mathbb{P}(I_i=0)=\mathbb{P}\left(min\{\lambda_i,\mu_i\}=\mu_i\right)=\frac{\mu_i}{\lambda_i+\mu_i}
\end{cases}\]
Mentre:
\[\begin{cases}
\mathbb{E}(T_i|I_i=1)=\frac{1}{\lambda_i+\mu_i} & \text{Basta far passare il tempo minimo tra i due} \\
\mathbb{E}(T_i|I_i=0)=\frac{1}{\lambda_i+\mu_i}+\mathbb{E}T_{i-1}+\mathbb{E}T_i & \text{Si scende e bisogna risalire: si sommano i tempi}
\end{cases}\]
Dunque in definitiva, nel nostro caso:
\[\mathbb{E}T_i=\mathbb{E}\left[\mathbb{E}\left(T_i|I_i\right)\right]=\left(\frac{1}{i+1}\right)^2+\frac{i}{i+1}\left[\frac{1}{i+1}+\mathbb{E}T_{i-1}+\mathbb{E}T_i\right]\]
\end{example}

Ultimo esempio di processo nascita morte è la coda M/M/s. 

\begin{example}Si considerino $s$ sportelli che hanno tasso di risoluzione  $\mu$ (liberano il cliente) indipendentemente, e un arrivo a tasso $\lambda$. Dunque si può modellare il processo come una processo nascita-morte con:
\[\begin{cases}
\lambda_n=\lambda \\
\begin{cases}
\mu_n=n\mu & n\leq s \\
\mu_n=s\mu & n>s
\end{cases}
\end{cases}\]
Questo è vero poichè se $n\leq s$ il tempo di morte (liberazione di uno sportello) è il minimo tra gli $n$ sportelli, dunque il minimo di esponenziali IID di parametro $\mu$. Se $n>s$ il tempo di morte è il minimo tra tutti gli gli sportelli, quindi tra gli $s$ sportelli.
\end{example}
\chapter{Moti Browniani}
\section{Nascita moto Browniano}
Iniziamo considerando il cammino casuale unidimensionale simmetrico $p=1/2$. Vogliamo ora adottare una nuova prospettiva sul problema, vederlo più da "\textit{lontano}" riscalandolo tramite intervalli di tempo e spazio sempre più piccoli, infinitesimi. Passando poi al limite otteremo un primo esempio di moto Brownian.

Sia $X(t)$ la posizione al tempo $t$:
\[X(t)=\Delta x\left(X_1+\dots+X_{\lfloor \frac{t}{\Delta t} \rfloor}\right)\]
Dove $X_i$ vale più o meno 1 a seconda che il passo $\Delta x$ sia a destra o a sinistra. Tutto il processo è riscalato ad intervallini $\Delta t$. Dunque si ottiene:
\begin{itemize}
    \item $\mathbb{E}X(t)=0$
    \item $\mathbb{V}arX(t)=\left(\Delta x\right)^2\lfloor \frac{t}{\Delta t} \rfloor$
\end{itemize}

Ma se gli intervalli di spazio e tempo tendono a zero con lo stesso ordine si ottiene una varianza nulla, e dunque il processo si schiaccia sull'asse $x$. Non è evidentemente l'obiettivo di una buona modellizzazione. 

Einstein, in un celebre articolo del 1905, "suggerisce" un approccio diverso, nel quale tra spazio e tempo interrcorre una relazione del tipo:
\[\Delta x=\sigma\sqrt{\Delta t}\]
Così facendo si ottiene, tramite il teorema del limite centrale otteniamo:
\[X(t)\sim N(0,\sigma^2t)\]
Ecco che otteniamo un moto Browniano. 

Passiamo a una definizione rigorosa.

\begin{definition}
Un processo stocastico si definisce \textbf{moto Browniano} se:
\begin{enumerate}
    \item $X(0)=0$
    \item Il processo ha incrementi indipendenti e stazionari
    \item $\forall t>0$ e per un parametro $\sigma^2$ si ha:
    \[X(t)\sim N(0,\sigma^2t)\]
    \item Le traiettorie del processo sono continue
\end{enumerate}
\end{definition}
In realtà la condizione (4.) può essere dimostrata dalle altre. Un moto Browniano ha la peculiare proprietà di essere continuo ovunque ma non differenziabile in nessun punto.

La condizione (3.) con la stazionarietà degli incrementi è equivalente a: 
\[X(t)-X(s)\sim N(0,\sigma^2(t-s)) \hspace{7px} \forall t,s\]
La condizione (3.) ci fornisce inoltre un importantissima informazione, la distribuzione del moto in un dato momento, in particolare:
\[f_t(x)=\frac{1}{\sqrt{2\pi\sigma^2 t}}e^{-\frac{x^2}{2\sigma^2t}}\]

Osservando che:
\[\{X(t_1)=x_1,X(t_2)=x_2,\dots,X(t_n)=x_n\}=\]
\[\{X(t_1)=x_1-\underbrace{x_0}_{=0},X(t_2)-X(t_1)=x_2-x_1,\dots,X(t_n)-X(t_{n-1})=x_n-x_{n-1}\}\]
Posso calcolare la densità congiunta (di un moto Browniano tandard, cioè con $\sigma=0$), ricordandomi dell'indipendenza degli intervalli disgiunti:
\[f(x_1,\dots,x_n)=\frac{1}{(2\pi)^{\frac{n}{2}}[t_1\dots(t_n-t_{n-1})]}e^{-\frac{1}{2}\left[\frac{x_1^2}{t_1}+\dots+\frac{(x_n-x_{n-1})^2}{t_n-t_{n-1}}\right]}\]

\begin{example}[Ponte Browniano]
Dato un moto browniano standard $\{X(t)\}$ e un $t\in\mathbb{R}_+$ fissato tale che $X(t)=B$, per $s<t$ calcolo:
\[f_{s|t}(x|B)=\frac{f_{s,t}(x,B)}{f_{s}(x)}=\frac{f_s(x)f_{t-s}(x-B)}{f_t(B)}=K_1exp\left\{-\frac{x^2}{2s}-\frac{(x-B)^2}{2(t-s)}\right\}\]
\[=K_2exp\left\{-x^2\left(\frac{1}{2s}+\frac{1}{2(t-s)}\right)+\frac{xB}{t-s}\right\}=K_3exp\left\{\frac{-t}{2s(t-s)}\left(x^2-\frac{2xsB}{t}\right)\right\}\]
\[=K_4exp\left\{\frac{-t}{2s(t-s)}\left(x-\frac{s}{t}B\right)^2\right\}\]
($K_i$ sono indipendenti da $s$).
\newline Ottenendo dunque che la distribuzione di probabilità in un ponte browniano è una normale di media e varianza:
\begin{itemize}
    \item $\mathbb{E}[X(s)|X(t)=B]=\frac{s}{t}B$
    \item $\mathbb{V}ar[X(s)|X(t)=B]=\frac{s}{t}(t-s)$
\end{itemize}
\end{example}

Altro esempio degno di nota, che richiama la rovina del giocatore (\ref{Rov_gioc}) è il seguente.

\begin{example}
Sia $T_a=Inf\{t>0:X(t)\geq a\}$ e studiamone la distribuzione:
\[\mathbb{P}(X(t)\geq a)=\mathbb{P}(X(t)\geq a|T_a\leq t)\mathbb{P}(T_a\leq t)+\mathbb{P}(X(t)\geq a|T_a>t)\mathbb{P}(T_a>t)\]
Si nota subito che:
\[\mathbb{P}(X(t)\geq a|T_a > t)=0\]
E inoltre:
\[\mathbb{P}(X(t)\geq a|T_a\leq t)=\frac{1}{2}\]
Questo secondo risultato deriva dal fatto che per simmetria ogni traiettoria sopra ad $a$ ce n'è una sotto.

Si ottiene così:
\[\mathbb{P}(T_a\leq t)=2\mathbb{P}(X(t)\geq a)\]
Dunque:
\[f_{T_a}(t)=\frac{d}{dt}\left[2\int_a^{+\infty}\frac{1}{\sqrt{2\pi\sigma^2 t}}e^{-\frac{x^2}{2\sigma^2t}} \,dx\right]\underset{y=\frac{x}{\sqrt{\sigma^2t}}}{=}\frac{d}{dt}\left[\sqrt{\frac{2}{\pi}}\int_{a/\sqrt{\sigma^2t}}^{+\infty}e^{-\frac{y^2}{2}} \,dy\right]=\]
\[\frac{a}{\sqrt{2\pi\sigma^2t^3}}e^{\frac{-a^2}{2\sigma^2t}}\]
\end{example}
 Da questo risultato si può calcolare la distribuzione di $M:=max_{s=0,\dots,t}\{X(s)\}$. Infatti:
 \[\frac{d}{da}\mathbb{P}(M\geq a)=\frac{d}{da}\left(\mathbb{P}(T_a\leq t)\right)=\frac{d}{da}\left(2\int_a^{+\infty}\frac{1}{\sqrt{2\pi\sigma^2 t}}e^{-\frac{x^2}{2\sigma^2t}} \,dx\right)=\sqrt{\frac{2}{\pi\sigma^2t}}e^{-\frac{a^2}{2\sigma^2t}}\]
\subsection{Tipologie di Moti Browniani}
\begin{definition}
Un processo stocastico $\{X(t)\}$ è un \textbf{moto browniano con drift} se:
\begin{itemize}
    \item $X(0)=0$
    \item Ha incrementi stazionari e indipendenti
    \item $X(t)\sim N(\mu t,\sigma^2 t)$
\end{itemize}
\end{definition}
Un moto browniano con drift può anche essere visto come la composizione lineare di un moto browniano standard e un drift $\mu t$ del tipo:
\[X(t)=\sigma B(t)+\mu t\]

\begin{definition}
Dato un moto browniano con drift $\{Y(t)\}$ e coefficiente di drift $\mu$ si definisce \textbf{moto browniano geometrico} il seguente:
\[X(t)=e^{Y(t)}\]
\end{definition}

Calcoliamo ora il valore atteso di un moto browniano geometrico dat la sua storia siano ad un tempo stabilito:
\[\mathbb{E}[X(t)|X(u), 0\leq u\leq s] \hspace{25px} \text{per} \hspace{5px} s<t\]
Per il calcolo si noti che conoscere $Y(u)$ o conoscere $e^{Y(u)}$ è la stessa cosa.
\[\mathbb{E}[X(t)|X(u), 0\leq u\leq s]=\mathbb{E}[e^{Y(t)}|Y(u), 0\leq u\leq s]=X(s)\mathbb{E}[e^{Y(t)-Y(s)}|Y(u), 0\leq u\leq s]\]
Ricordando che la funzione caratteristica di una normale è della forma $e^{\mu t+\frac{\sigma^2t^2}{2}}$ (\ref{char_N}):
\[\mathbb{E}[X(t)|X(u), 0\leq u\leq s]=X(s)e^{\mu(t-s)+\frac{\sigma^2}{2}(t-s)}\]
Se $\mu=-\frac{\sigma^2}{2}$ si ottiene la \href{https://en.wikipedia.org/wiki/Martingale_(probability_theory)}{proprietà martingale}.

Come ultimo spunto siano ad esempio $X_n$ i prezzi di un'azione al tempo $n$, sembra ragionevole considerare la variazione percentuale $Y_n=\frac{X_n}{X_{n-1}}$ come IID. Si verifichi (pag. 645 - S.M.Ross) che $\{X_n\}$ è un moto browniano geometrico (\textit{Hint: si scriva $X_n$ in funzione delle $Y_i$ e si usi il teorema ddel limite centrale}).
\end{document}