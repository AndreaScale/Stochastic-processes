\section{processi di conteggio e di Poisson}
Introduciamo ora una nuova famiglia di processi, i \textbf{processi di conteggio}. I processi di conteggio, come suggerisce il nome, contano il numero di eventi che si verificano in un dato intervallo di tempo. Si passa dunque al continuo per l'insieme indice.
\begin{definition}
Il processo stocastico $\{N(t)\}_{t\geq 0}$ è di \textbf{conteggio} se:
\begin{itemize}
    \item $N(t)\geq 0$
    \item $N(t)\in\mathbb{N}$
    \item $N(t)\leq N(s)$ per $t\leq s$
    \item La quantità $N(t)-N(s)$ detta \textbf{incremento} conta il numero di eventi nell'intervallo di tempo $(s,t]$
\end{itemize}
Un processo è a \textbf{incrementi indipendenti} se il numero di eventi che si verificano su intervalli disgiunti sono variabili aleatorie indipendenti. Un processo è a \textbf{incrementi stazionari} se il numero di eventi che avvengono in un lasso di tempo dipende solo dalla lunghezza di esso, e non dal momento iniziale: \[N(t+s)-N(s)\sim N(t)-N(0)\]
$\forall s,t\in\mathbb{R}_+$.
\end{definition}
Così definiti i processi di conteggio possiamo definire quelli di Poisson.
\begin{definition}[Poisson 1]
Un processo di conteggio $\{N(t)\}_{t\geq0}$ è di \textbf{Poisson} se:
\begin{itemize}
    \item $N(0)=0$
    \item A incrementi indipendenti
    \item A incrementi stazionari (\textit{condizione sovrabbondante. Si può derivare dalle altre.})
    \item $\mathbb{P}(N(t)-N(s)=k)=\frac{(\lambda(t-s))^k}{k!}e^{-\lambda(t-s)}$
\end{itemize}
\end{definition}

Il processo di Poisson può essere definito in altri modi equivalenti, facendo uso degli incrementi possibili in un lasso di tempo infinitesimo. In particolare in $[t,t+h)$ se $h$ è infinitesimo potrà verificarsi al più un evento, quindi:
\[\mathbb{P}(N(t+h)-N(t)=1)=\lambda he^{-\lambda h}=\lambda h(1-\lambda h + o(h))=\lambda h + o(h)\]
\[\mathbb{P}(N(t+h)-N(t)=0)=1-\lambda h+o(h)\]
Dunque si può dare l'equivalente definizione:
\begin{definition}[Poisson 2]
Un processo di conteggio $\{N(t)\}_{t\geq0}$ è di \newline \textbf{Poisson} se:
\begin{itemize}
    \item $N(0)=0$
    \item A incrementi indipendenti e stazionari
    \item $\mathbb{P}(N(h)=1)=\lambda h+o(h)$
    \item $\mathbb{P}(N(h)\geq2)=o(h)$
\end{itemize}
\end{definition}
Si mostra che le due definizioni sono equivalenti:
\begin{itemize}
    \item $1\Rightarrow 2$: è gia stato dimostrato sopra.
    \item $2\Rightarrow 1$: Sia $u\geq0$ fissato e considero la trasformata di Laplace:
    \[g(u)=\mathbb{E}\left[e^{-uN(t)}\right]\]
    Calcolo:
    \[g(t+h)=\mathbb{E}\left[e^{-uN(t+h)}\right]=\mathbb{E}\left[e^{-u(N(t+h)-N(t)+N(t))}\right]\]
    Per indipendenza e stazionarietà del processo:
    \[g(t+h)=\mathbb{E}\left[e^{-uN(h)}\right]\mathbb{E}\left[e^{-uN(t)}\right]=g(t)\mathbb{E}\left[e^{-uN(h)}\right]\]
    Dunque calcolo:
    \[\mathbb{E}\left[e^{-uN(h)}\right]=e^{-u\cdot0}(1-\lambda h +o(h))+e^{-u\cdot 1}(\lambda h+o(h))+o(h)=\]
    \[1-\lambda h+e^{-u}\lambda h+ o(h)\]
    Ottenendo:
    \[g(t+h)-g(t)=g(t)\big(h\lambda(e^{-u}-1)+o(h)\big)\]
    E dunque:
    \[g'(t)=\lim_{h\to0}\frac{g(t+h)-g(t)}{h}=g(t)\lambda(e^{-u}-1)\]
    Ottenendo infine:
    \[\int_0^t\frac{dg}{g}=\int_0^t\lambda(e^{-u}-1) \,dt \Rightarrow g(t)=g(0)e^{\lambda t(e^{-u}-1)}\]
    Che è la trasformata di Laplace di una Poisson di parametro $\lambda t$.
\end{itemize}
\subsection{Intertempi}
\vspace{5px} 
Un importante fattore caratterizzante per un processo di Poisson sono gli \textbf{intertempi} tra eventi, cioè:
\begin{center}
$T_i$ = Tempo attesa dall'evento $(i-1)$-esimo all'$i$-esimo
\end{center}
Studiamone la distribuzione, osservando che $\{T_1>t\}=\{N(t)=0\}$ come eventi:
\[\mathbb{P}(T_1>t)=\mathbb{P}(N(t)=0)=e^{-\lambda t}\]
Equivalentemente:
%osservando che $\{T_2>t|T_1=s\}=\{N(t+s)-N(s)=0\}=\{N(t)=0\}$, si ottiene:
\[\mathbb{P}(T_2>t|T_1=s)=\frac{\mathbb{P}(T_2>t,T_1=s)}{\mathbb{P}(T_1=s)}=\frac{\mathbb{P}(N(t+s)-N(s)=0,N(s)=1)}{\mathbb{P}(N(s)=1)}\]
Poichè gli incrementi sono indipendenti si riduce:
\[\mathbb{P}(N(t+s)-N(s)=0)=e^{-\lambda t}\]
%Ora si usa questo risultato in:
%\[\mathbb{P}(T_2>t)=\mathbb{E}(\mathbb{P}(T_2>t|T_1))=\int_0^tse^{-\lambda t} \,ds\]
%\[(N(t)=0)=e^{-\lambda t}\]
Dunque, procedendo iterativamente (osservando che $\mathbb{P}(T_2>t|T_1=s)$ \textit{non} dipende da $s$) gli intertempi sono v.a. IID di distribuzione $Exp(\lambda)$. Grazie a questa caratterizzazione si può dare una terza definizione equivalente di Processo di Poisson.
\begin{definition}[Poisson 3]
Una processo di conteggio con intertempi IID come esponenziali di parametro $\lambda$ è un processo di Poisson di parametro $\lambda$.
\end{definition}

\begin{observation}
Il processo di Poisson è \textit{markoviano} con matrice di transizione:
\[P=\begin{bmatrix}
0 & 1 & 0 & \dots \\
0 & 0 & 1 & 0 \\
0 & 0 & 0 & 1
\end{bmatrix}\]
Questo è ovvio poichè il processo può solo salire, e questo evviene con probabilità $1$. In quanto tempo venga invece è caratterizzato dal parametro della Poisson in considerazione.
\end{observation}

\subsection{Ulteriori proprietà del processo di Poisson}
Sia $\{N(t)\}$ un processo di Poisson. Ogni evento è suddiviso tra tipo I e tipo II, definiamo $N_1(t)$ e $N_2(t)$ i processi di conteggio che contano rispettivamente i tipi I e II. Si verifica che $\{N_1(t)\},\{N_2(t)\}$ sono due processi di Poisson di parametri $\lambda p$ e $\lambda(1-p)$:
\begin{itemize}
    \item $N_1(0)=0$ poichè $N(0)=0$ 
    \item $N_1$ eredita incrementi indipendenti e stazionari
    \item \begin{itemize}
        \item \[\mathbb{P}(N_1(h)=1)=\]
        \[\mathbb{P}(N_1(h)=1|N(h)=1)\mathbb{P}(N(h)=1)+\mathbb{P}(N_1(h)=1|N(h)\geq2)\mathbb{P}(N(h)\geq2)\]
        \[p(\lambda h+o(h))+o(h)=\lambda ph+o(h)\]
        \item 
        \[\mathbb{P}(N_1(h)\geq2)\leq\mathbb{P}(N_1\geq2)=o(h)\]
    \end{itemize}
\end{itemize}
Equivalentemente si dimostra per $N_2(t)$.\vspace{10px}

Consideriamo due processi di Poisson indipendenti e $S_n^1$ e $S_m^2$ i tempi dell'$n$-esimo e $m$-esimo evento rispettivamente per il processo $1$ e per il $2$. Vogliamo studiare $\mathbb{P}(S_n^1<S_m^2)$.
\begin{itemize}
    \item $m=n=1$: \[\mathbb{P}(S_n^1<S_m^2)=\mathbb{P}(T_1<T_2)=\frac{\lambda_1}{\lambda_1+\lambda_2}\]
    \item $n=2,m=1$: \[\mathbb{P}(S_n^1<S_m^2)=\mathbb{P}(S_2^1<S_1^2|S_1^1<S_1^2)\mathbb{P}(S_1^1<S_1^2)=\]
    \[\mathbb{P}(S_1^1<S_1^2)\mathbb{P}(S_1^1<S_1^2)=\left(\frac{\lambda_1}{\lambda_1+\lambda_2}\right)^2\]
\end{itemize}
%Il ragionamento applicato sfrutta la mancanza di memoria. 
Ogni volta che si verifica uno dei due processi la probabilità che il prossimo a verificarsi di $1$ è $p_1=\frac{\lambda_1}{\lambda_1+\lambda_2}$ mentre che sia $2$ è $p_2=\frac{\lambda_2}{\lambda_1+\lambda_2}$. Così ragionando si ottiene:
\[\mathbb{P}(S_n^1<S_m^2)=\sum_{k=n}^{n+m-1}\binom{n+m-1}{k}\left(\frac{\lambda_1}{\lambda_1+\lambda_2}\right)^k\left(\frac{\lambda_2}{\lambda_1+\lambda_2}\right)^{n+m-1-k}\]

\subsection{Distribuzione condizionale dei tempi di arrivo}

Viene chiesto di determinare la distribuzione del tempo necessario affinchè un evento in un processo di Poisson si verifichi, sapendo che un evento si è verificato in $[0,t]$. Poichè il conteggio di Poisson e a incrementi stazionari e indipendenti il tempo di realizzazione deve essere uniformemente distribuito su $[0,t]$:
\[\mathbb{P}(T_1<s|N(t)=1)=\frac{\mathbb{P}(T_1<s,N(t)=1)}{\mathbb{P}(N(t)=1)}=\frac{\mathbb{P}(N_{[0,s)}=1,N_{[s,t]}=0)}{\mathbb{P}(N(t)=1)}=\]
\[\frac{\mathbb{P}(N_{[0,s)}=1)\mathbb{P}(N_{[s,t]}=0)}{\mathbb{P}(N(t)=1)}=\frac{\lambda se^{-\lambda s}}{\lambda te^{-\lambda t}}e^{-\lambda(t-s)}=\frac{s}{t}\]

Introduciamo il concetto di statistiche d'ordine.
\begin{definition}
Siano $\{Y_n\}$ v.a. IID si dice che $Y_{(1)},...,Y_{(n)}$ sono le \textbf{statistiche d'ordine} corrispondenti se a $Y_{(k)}$ corrisponde il $k$-esimo valore più piccolo tra le v.a.
\end{definition}

\begin{proposition}
Date $\{Y_n\}$ v.a.  assolutamente continue IID di densità $f$, allora la densità della statistica d'ordine è:
\[\overline{f}(y_1,...,y_2)=n!\prod_{i=1}^nf(y_i)\]
\begin{proof}
$Y_{(1)},...,Y_{(n)}$ è uguale a $y_{1},...,y_{n}$ se $\{Y_n\}$ è uguale a una delle $n!$ permutazioni di $y_{1},...,y_{n}$. Ora poichè $f(y_1,...,y_2)=\prod_{i=1}^nf(y_i)$ allora si ottiene la tesi.
\end{proof}
\end{proposition}

Ora, se $Y_n$ sono uniformemente distribuite su $[0,t]$ allora: \[\overline{f}(y_1,...,y_n)=\frac{n!}{t^n}\]
Ciò ci porta a:
\begin{theorem}
I tempi di arrivo dato $N(t)=n$ hanno la stessa distribuzione della statistica ordinata corrispondente a $n$ v.a. IID uniformemente distribuite su $[0,t]$.
\begin{proof}
Si osserva che per $0\geq s_1\geq...\geq s_n$ vale l'uguaglianza di eventi:
\[\left\{S_1=s_1,...,S_n=s_n,T(t)=n\right\}=\left\{T_1=s_1,...,T_n=s_n-s_{n-1},T_{n+1}>t-s_n\right\}\]
Dunque:
\[f(s_1,...,s_n|n)=\frac{f(s_1,...,s_n,n)}{\mathbb{P}(T(t)=n)}=\frac{\lambda e^{-\lambda s_1}\dots\lambda e^{-\lambda (s_n-s_{n-1})}e^{-\lambda (t-s_n)}}{\frac{\left(\lambda t\right)^n}{n!}e^{-\lambda t}}=\frac{n!}{t^n}\]
\end{proof}
\end{theorem}

\subsection{Generalizzazioni di processi di Poisson}
\subsubsection{Porcesso di Poisson nonomogeneo}
\begin{definition}[Poisson 2]
Un processo di conteggio $\{N(t)\}_{t\geq0}$ è di \newline \textbf{Poisson nonomogeneo} con \textbf{funzione d'intensità} $\lambda(t)$ se:
\begin{itemize}
    \item $N(0)=0$
    \item A incrementi indipendenti 
    \item $\mathbb{P}(N(t+h)-N(t)=1)=\lambda(t) h+o(h)$
    \item $\mathbb{P}(N(t+h)-N(t)\geq2)=o(h)$
\end{itemize}
Si definisce inoltre la \textit{funzione del valore atteso di una Poisson nonomogenea}:
\[m(t)=\int_0^t\lambda(y) \,dy\]
\end{definition}

Si dimostra che:
\[\mathbb{P}(N(t+s)-N(t)=n)=e^{m(t+s)-m(t)}\frac{[m(t+s)-m(t)]^n}{n!}\]

\subsubsection{Porcesso di Poisson composto}

Un processo di Poisson si dice \textbf{composto} se è rappresentabile come:
\[X=\sum_{i=1}^{N(t)}Y_i\]
Dove $Y_i$ sono v.a. IID e $N(t)$ un processo di Poisson indipendente dalle v.a.